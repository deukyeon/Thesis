%% Replace all of the orange portions with your personal info

\thispagestyle{empty}
\begin{centering}
\vspace{1in}
University of Washington \\
\vspace*{1.\baselineskip}
{\bf Abstract}\\
\vspace*{1\baselineskip}

{\thesisTitle}\\ %self-explanatory
\vspace*{1.\baselineskip}
{\authorName} \\ %self-explanatory
\vspace*{1.\baselineskip}


\ifdefined\secondAdvisor
    Co-chairs
    \else
    Chair
\fi
of the Supervisory Committee:\\ %change to co-chair if co-advised 
%\advisorTitle~
\advisor\\ \vspace{-.5em} \advisorDepartment \\
\ifdefined\secondAdvisor
    \secondAdvisorTitle~\secondAdvisor\\\vspace{-.5em}\secondAdvisorDepartment \\
\fi
\end{centering}
\vspace*{\baselineskip}

Achieving high-performance transaction processing in disk-based databases has
long required system designers to choose between lock-based concurrency control
methods, which suffer from CPU overhead and reduced parallelism, and
timestamp-based methods, which provide superior concurrency but incur
prohibitive I/O overhead when timestamp metadata is stored on disk. Modern
high-speed storage devices like NVMe SSDs exacerbate this trade-off, as the
CPU becomes the bottleneck for lock-based methods while disk-based timestamp
storage wastes the storage device's speed on frequent small metadata
operations.

This dissertation introduces a novel approach that eliminates this fundamental
trade-off through approximate timestamp storage and demonstrates that
timestamp-based concurrency control protocols---specifically Strict Timestamp
Ordering (STO), Multi-Version Timestamp Ordering (MVTO), and TicToc---can
maintain correctness (serializability) even when timestamps are overapproximated
for inactive keys, as long as active keys maintain exact timestamps throughout
their transaction lifetime. This key insight enables designing \sketchname, a
hybrid data structure combining a hash table for exact timestamps of active keys
with a probabilistic sketch for approximate upper bounds of inactive keys.

The first contribution is the design, implementation, and evaluation of
\sketchname integrated with STO, MVTO, and TicToc in the SplinterDB key-value
store. \sketchname achieves nearly the idealized performance while requiring only
minimal memory---as little as 32KiB for an 80GB database---by eliminating the
need to access timestamp metadata from disk during normal operation.
Experimental evaluation on modern NVMe SSDs demonstrates that TicToc with
\sketchname achieves up to 14$\times$ higher goodput than traditional two-phase
locking, up to 5.9$\times$ higher goodput than disk-based timestamp storage.

The second contribution is a comprehensive analytical and experimental study
evaluating \sketchname across the entire storage performance spectrum, from
traditional hard disk drives with millisecond latencies to emerging CXL-based
storage approaching DRAM-like speeds. The evaluation reveals that \sketchname's
benefits scale with the fundamental gap between local memory and remote storage
access, ensuring its continued relevance as storage technology evolves. On slow
storage (HDDs and SATA SSDs), \sketchname enables timestamp-based protocols to
outperform traditional concurrency control methods: on SATA SSD,
TicToc-\psketchname achieves up to 6.89$\times$ and 2.52$\times$ higher goodput
than two-phase locking (2PL) and KR-OCC, respectively, while on HDD it reaches
up to 1.8$\times$ the goodput of KR-OCC. \sketchname also eliminates the
prohibitive overhead of timestamp disk accesses, achieving improvements of up to
569\% over disk-based timestamp storage. On fast storage (simulated CXL-based
SSDs), where systems transition from I/O-bound to CPU-bound, \sketchname
continues to provide substantial benefits by keeping timestamp metadata in fast
local memory, enabling timestamp-based protocols to significantly outperform
traditional approaches.

Together, these contributions establish that approximate, in-memory metadata
management enables high-performance transaction processing for disk-based
databases. \sketchname demonstrates that approximate metadata management can
unlock advanced concurrency control designs that would otherwise be impractical,
providing a practical solution that enables efficient timestamp-based
concurrency control across diverse storage technologies while requiring only
minimal memory overhead.

