Storage technologies in production environments span several orders of magnitude
in performance---from traditional hard disk drives (HDDs) with millisecond 
latencies to emerging byte-addressable storage like CXL-based flash approaching 
DRAM-like speeds. Since \sketchname enables modern timestamp-based concurrency 
control for disk-based key-value stores, evaluating its effectiveness across this 
performance spectrum is critical for understanding both current deployment 
opportunities and future applicability as storage technologies continue to evolve.

This chapter presents a comprehensive evaluation across slow storage (SATA SSD and 
HDD) in Section~\ref{sec:slow-storage} and fast storage (emulated CXL-based flash) 
in Section~\ref{sec:fast-storage} to reveal how \sketchname's benefits and 
trade-offs evolve with storage characteristics. Our analysis demonstrates that 
\sketchname remains effective across this wide range, though the nature of its 
advantages changes fundamentally as storage performance improves.


\section{Slow Storage}\label{sec:slow-storage}

We begin our evaluation with slow storage scenarios, which represent traditional
disk-based systems where storage latency significantly impacts transaction
processing performance. On slow storage, disk I/O becomes the dominant bottleneck,
making the elimination of timestamp disk accesses provided by \sketchname 
particularly valuable. These results demonstrate how \sketchname performs under
challenging storage conditions and reveal its effectiveness when storage 
characteristics fundamentally limit transaction throughput.

We use the same workloads and evaluation methods as in Section~\ref{sec:eval}.
For slow storage setups, we run experiments using both SATA SSD and HDD. The
SATA SSD experiments are run on the same type of machine as the NVMe SSD tests,
with only the storage hardware changed. However, for the HDD tests, we use a
different machine because CloudLab does not offer machines with HDDs. Using SATA
SSD allows us to compare results fairly with NVMe SSD and other faster storage.
In contrast, the HDD experiments show how \sketchname performs on a real hard
disk, helping us understand its behavior in truly slow storage situations.


\subsection{SATA SSD}

\subsubsection{YCSB Small-Transaction Workloads}

Figure~\ref{fig:ycsb:slow_ssd} presents the goodput of all our timestamp-based
concurrency control (CC) variants on four YCSB small-transaction workloads using
slow SSD storage. The patterns in the results are similar to those shown in
Section~\ref{sec:eval} for NVMe SSD. The \sketchname variants regularly achieve
goodput very close to the Memory setup, which shows that they remain effective
even when storage is slower.

In read-intensive workloads, the \sketchname variants (\fsketchname and
\psketchname) almost match the goodput of the Memory setup. They improve goodput
by as much as 39\% compared to the Disk variant when contention is high, and up
to 23\% better in medium-contention situations. The TicToc-\psketchname setup in
particular improves goodput by 65\% over 2PL and by 19\% over KR-OCC in
workloads with high contention.

For write-intensive workloads, the difference in goodput becomes even greater.
Here, the \sketchname variants (\fsketchname and \psketchname) again produce
goodput close to the Memory variant. In high-contention cases, they boost
goodput by up to 569\% over the Disk variant, and by up to 119\% when contention
is medium. The TicToc-\psketchname configuration shows especially strong
results, delivering 589\% and 152\% higher goodput than 2PL and KR-OCC,
respectively, for high-contention workloads.

TicToc is a highly concurrent CC method that uses timestamps. When combined with
\sketchname, TicToc achieves the best performance in all workloads. This is
because \sketchname removes the need to frequently access timestamps on disk,
which lessens storage delays. \sketchname is especially helpful for
write-intensive, high-contention workloads, because these workloads consist of
quick operations enabled by SplinterDB's fast write speed and high cache hit
rate even though storage is slow.



\begin{figure}[!t]
    \centering
    \resizebox{\textwidth}{!}{%
        \begin{tikzpicture}
            \begin{groupplot}[group style={group size=4 by 2,
                            /pgf/bar width=2.7pt},
                    height = 3.6cm,
                    width = 4.5cm,
                    ybar= 2*\pgflinewidth,
                    xtick={-1, 0.425, 1.7125, 3.1},
                    xticklabels={2PL/OCC, STO, MVTO, TicToc},
                    ymajorgrids=true,
                    enlarge x limits = 0.2,
                    legend columns=-1,
                    legend entries={{\ssmall Memory (Idealized)}, {\ssmall Disk}, {\ssmall Disk-Cache}, {\ssmall \fsketchname},  {\ssmall \psketchname}},
                    area legend,
                    legend to name=grouplegend,
                ]

                % graph [1,1] high
                \nextgroupplot[title={Read-intensive},YCSBThroughputBarChartPlot, ylabel={Goodput (KTPS)}]
                \addplot[2PL-BarStyle, forget plot] coordinates {(-0.5, 10.364933)};
                \addplot[KR-OCC-BarStyle, forget plot] coordinates {(-0.26, 14.441267)};
                \addplot[Memory-BarStyle] coordinates {(0.425, 14.425833) (1.7125, 13.695567) (3, 16.9643)};
                \addplot[Disk-BarStyle] coordinates {(0.425, 12.791433) (1.7125, 0.05428) (3, 12.298733)};
                \addplot[Disk-Cache-BarStyle] coordinates {(0.425, 9.92869) (1.7125, 9.340693) (3, 11.959133)};
                \addplot[Counter-Lazy-BarStyle] coordinates {(0.425, 14.190467) (1.7125, 13.124533) (3, 16.308867)};
                \addplot[FPSketch-BarStyle] coordinates {(0.425, 14.404033) (1.7125, 13.169733) (3, 17.120333)};
                \coordinate (top) at (rel axis cs:0,1);% coordinate at top of the first plot

                % graph [1,3] high
                \nextgroupplot[title={Write-intensive}, YCSBThroughputBarChartPlot]
                \addplot[2PL-BarStyle, forget plot] coordinates {(-0.5, 5.088527)};
                \addplot[KR-OCC-BarStyle, forget plot] coordinates {(-0.26, 13.9281)};
                \addplot[Memory-BarStyle] coordinates {(0.425, 7.07645) (1.7125, 5.59584) (3, 39.2637)};
                \addplot[Disk-BarStyle] coordinates {(0.425, 4.271767) (1.7125, 0.05016) (3, 5.239043)};
                \addplot[Disk-Cache-BarStyle] coordinates {(0.425, 3.879563) (1.7125, 3.21073) (3, 4.171167)};
                \addplot[Counter-Lazy-BarStyle] coordinates {(0.425, 7.14758) (1.7125, 5.09302) (3, 28.504433)};
                \addplot[FPSketch-BarStyle] coordinates {(0.425, 7.34722) (1.7125, 5.753657) (3, 35.0623)};

                % graph [1,1] medium
                \nextgroupplot[title={Read-intensive},YCSBThroughputBarChartPlot]
                \addplot[2PL-BarStyle, forget plot] coordinates {(-0.5, 5.74492)};
                \addplot[KR-OCC-BarStyle, forget plot] coordinates {(-0.26, 5.701743)};
                \addplot[Memory-BarStyle] coordinates {(0.425, 5.749967) (1.7125, 5.49676) (3, 5.710213)};
                \addplot[Disk-BarStyle] coordinates {(0.425, 5.171957) (1.7125, 0.046666) (3, 4.639907)};
                \addplot[Disk-Cache-BarStyle] coordinates {(0.425, 4.499663) (1.7125, 4.262087) (3, 4.50323)};
                \addplot[Counter-Lazy-BarStyle] coordinates {(0.425, 5.717253) (1.7125, 5.33784) (3, 5.73195)};
                \addplot[FPSketch-BarStyle] coordinates {(0.425, 5.67032) (1.7125, 5.405587) (3, 5.707593)};


                % graph [1,3] medium
                \nextgroupplot[title={Write-intensive},YCSBThroughputBarChartPlot]
                \addplot[2PL-BarStyle, forget plot] coordinates {(-0.5, 9.648553)};
                \addplot[KR-OCC-BarStyle, forget plot] coordinates {(-0.26, 9.156227)};
                \addplot[Memory-BarStyle] coordinates {(0.425, 9.86155) (1.7125, 8.966833) (3, 9.90628)};
                \addplot[Disk-BarStyle] coordinates {(0.425, 4.785737) (1.7125, 0.044312) (3, 4.43323)};
                \addplot[Disk-Cache-BarStyle] coordinates {(0.425, 4.431213) (1.7125, 3.680533) (3, 4.480477)};
                \addplot[Counter-Lazy-BarStyle] coordinates {(0.425, 9.65755) (1.7125, 6.98781) (3, 9.650477)};
                \addplot[FPSketch-BarStyle] coordinates {(0.425, 9.859053) (1.7125, 6.96129) (3, 9.68747)};
                \coordinate (bot) at (rel axis cs:1,0);% coordinate at bottom of the last plot

                % graph [1,2] read intensive
                \nextgroupplot[AbortRateBarChartPlot, ylabel={Abort rate}]
                \addplot[2PL-BarStyle, forget plot] coordinates {(-0.5, 0.304596)};
                \addplot[KR-OCC-BarStyle, forget plot] coordinates {(-0.26, 0.415079)};
                \addplot[Memory-BarStyle] coordinates {(0.425, 0.310117) (1.7125, 0.294705) (3, 0.13013)};
                \addplot[Disk-BarStyle] coordinates {(0.425, 0.309437) (1.7125, 0.285392) (3, 0.134926)};
                \addplot[Disk-Cache-BarStyle] coordinates {(0.425, 0.332255) (1.7125, 0.300908) (3, 0.130042)};
                \addplot[Counter-Lazy-BarStyle] coordinates {(0.425, 0.311321) (1.7125, 0.302894) (3, 0.157114)};
                \addplot[FPSketch-BarStyle] coordinates {(0.425, 0.308572) (1.7125, 0.304875) (3, 0.134233)};

                % graph [1,4] write intensive
                \nextgroupplot[AbortRateBarChartPlot]
                \addplot[2PL-BarStyle, forget plot] coordinates {(-0.5, 0.571121)};
                \addplot[KR-OCC-BarStyle, forget plot] coordinates {(-0.26, 0.495017)};
                \addplot[Memory-BarStyle] coordinates {(0.425, 0.520538) (1.7125, 0.54321) (3, 0.280598)};
                \addplot[Disk-BarStyle] coordinates {(0.425, 0.528094) (1.7125, 0.533668) (3, 0.272687)};
                \addplot[Disk-Cache-BarStyle] coordinates {(0.425, 0.523497) (1.7125, 0.551743) (3, 0.266006)};
                \addplot[Counter-Lazy-BarStyle] coordinates {(0.425, 0.51712) (1.7125, 0.551956) (3, 0.316003)};
                \addplot[FPSketch-BarStyle] coordinates {(0.425, 0.511699) (1.7125, 0.540079) (3, 0.289621)};

                % graph [1,2]
                \nextgroupplot[AbortRateBarChartPlot]
                \addplot[2PL-BarStyle, forget plot] coordinates {(-0.5, 0.000554)};
                \addplot[KR-OCC-BarStyle, forget plot] coordinates {(-0.26, 0.000204)};
                \addplot[Memory-BarStyle] coordinates {(0.425, 0.000093) (1.7125, 0.000098) (3, 0.000032)};
                \addplot[Disk-BarStyle] coordinates {(0.425, 0.000085) (1.7125, 0.000089) (3, 0.000037)};
                \addplot[Disk-Cache-BarStyle] coordinates {(0.425, 0.000089) (1.7125, 0.000077) (3, 0.000033)};
                \addplot[Counter-Lazy-BarStyle] coordinates {(0.425, 0.479464) (1.7125, 0.483513) (3, 0.000094)};
                \addplot[FPSketch-BarStyle] coordinates {(0.425, 0.001726) (1.7125, 0.000607) (3, 0.000087)};

                % graph [1,4]
                \nextgroupplot[AbortRateBarChartPlot]
                \addplot[2PL-BarStyle, forget plot] coordinates {(-0.5, 0.004177)};
                \addplot[KR-OCC-BarStyle, forget plot] coordinates {(-0.26, 0.000907)};
                \addplot[Memory-BarStyle] coordinates {(0.425, 0.000498) (1.7125, 0.000473) (3, 0.000191)};
                \addplot[Disk-BarStyle] coordinates {(0.425, 0.000566) (1.7125, 0.000276) (3, 0.000297)};
                \addplot[Disk-Cache-BarStyle] coordinates {(0.425, 0.000493) (1.7125, 0.000509) (3, 0.000351)};
                \addplot[Counter-Lazy-BarStyle] coordinates {(0.425, 0.553703) (1.7125, 0.566425) (3, 0.000453)};
                \addplot[FPSketch-BarStyle] coordinates {(0.425, 0.077443) (1.7125, 0.01577) (3, 0.000418)};

            \end{groupplot}
            \coordinate (c) at ($(top)!.5!(bot)$);
            \coordinate (cc1) at ($(top)!.25!(bot)$);
            \coordinate (cc2) at ($(top)!.75!(bot)$);

            \node[above] at (c |- current bounding box.north) {\ref{grouplegend}};
            \node[below] at (cc1 |- current bounding box.south) {(a) High-contended YCSB (zipf=0.99)};
            \node[above] at (cc2 |- current bounding box.south) {(b) Medium-contended YCSB (zipf=0.6)};
        \end{tikzpicture}
    } \caption[YCSB small-transaction workloads results on slow SSD]{YCSB
    small-transaction workloads results with 120 processing threads on slow SSD.
    In write-intensive workloads, each transaction performs 8 reads and 8
    writes. In read-intensive workloads, each transaction performs 15 reads and
    1 write. Note that, in the medium-contended workload, several variants have
    abort rates very close to 0.}
    \label{fig:ycsb:slow_ssd}
\end{figure}

\subsubsection{YCSB Mixed-Transaction Workloads}

\begin{figure}[!t]
    \centering
    \resizebox{\textwidth}{!}{%
        \begin{tikzpicture}
            \begin{groupplot}[group style={group size=4 by 1,
                            /pgf/bar width=2.7pt},
                    height = 3.6cm,
                    width = 4.5cm,
                    ybar= 2*\pgflinewidth,
                    xtick={-1, 0.425, 1.7125, 3.1},
                    xticklabels={2PL/OCC, STO, MVTO, TicToc},
                    ymajorgrids=true,
                    enlarge x limits = 0.225,
                    legend columns=-1,
                    legend entries={{\ssmall Memory (Idealized)}, {\ssmall Disk}, {\ssmall Disk-Cache}, {\ssmall \fsketchname},  {\ssmall \psketchname}},
                    area legend,
                    legend to name=grouplegend,
                ]

                % graph [1,1] high
                \nextgroupplot[YCSBThroughputBarChartPlot, ylabel={Goodput (KTPS)}, ylabel style={yshift=-5pt}]
                \addplot[2PL-BarStyle, forget plot] coordinates {(-0.5, 17.7908)};
                \addplot[KR-OCC-BarStyle, forget plot] coordinates {(-0.26, 13.9216)};
                \addplot[Memory-BarStyle] coordinates {(0.425, 21.6929) (1.7125, 19.823333) (3, 22.264467)};
                \addplot[Disk-BarStyle] coordinates {(0.425, 16.7041) (1.7125, 0.071738) (3, 13.9898)};
                \addplot[Disk-Cache-BarStyle] coordinates {(0.425, 13.890367) (1.7125, 11.8303) (3, 13.760167)};
                \addplot[Counter-Lazy-BarStyle] coordinates {(0.425, 18.941667) (1.7125, 15.2963) (3, 22.4423)};
                \addplot[FPSketch-BarStyle] coordinates {(0.425, 21.315967) (1.7125, 17.8994) (3, 22.084967)};
                \coordinate (top) at (rel axis cs:0,1);% coordinate at top of the first plot

                % graph [2,1] abort rate
                \nextgroupplot[AbortRateBarChartPlot, ylabel={Abort rate}, ylabel style={yshift=-3pt}]
                \addplot[2PL-BarStyle, forget plot] coordinates {(-0.5, 0.162546)};
                \addplot[KR-OCC-BarStyle, forget plot] coordinates {(-0.26, 0.281549)};
                \addplot[Memory-BarStyle] coordinates {(0.425, 0.123194) (1.7125, 0.050042) (3, 0.058888)};
                \addplot[Disk-BarStyle] coordinates {(0.425, 0.1363) (1.7125, 0.052943) (3, 0.057104)};
                \addplot[Disk-Cache-BarStyle] coordinates {(0.425, 0.139533) (1.7125, 0.052058) (3, 0.05523)};
                \addplot[Counter-Lazy-BarStyle] coordinates {(0.425, 0.170173) (1.7125, 0.125589) (3, 0.078445)};
                \addplot[FPSketch-BarStyle] coordinates {(0.425, 0.13226) (1.7125, 0.108899) (3, 0.06144)};

                % graph [3,1] medium
                \nextgroupplot[YCSBThroughputBarChartPlot, ylabel={Goodput (KTPS)}, ylabel style={yshift=-8pt}]
                \addplot[2PL-BarStyle, forget plot] coordinates {(-0.5, 11.919633)};
                \addplot[KR-OCC-BarStyle, forget plot] coordinates {(-0.26, 11.989233)};
                \addplot[Memory-BarStyle] coordinates {(0.425, 11.891433) (1.7125, 11.464433) (3, 11.743967)};
                \addplot[Disk-BarStyle] coordinates {(0.425, 9.385673) (1.7125, 0.082369) (3, 8.519537)};
                \addplot[Disk-Cache-BarStyle] coordinates {(0.425, 8.327213) (1.7125, 7.538987) (3, 8.14708)};
                \addplot[Counter-Lazy-BarStyle] coordinates {(0.425, 11.3492) (1.7125, 9.616147) (3, 11.706867)};
                \addplot[FPSketch-BarStyle] coordinates {(0.425, 11.7846) (1.7125, 10.483733) (3, 11.955533)};

                % graph [4,1] medium abort rate
                \nextgroupplot[AbortRateBarChartPlot, ylabel={Abort rate}, ylabel style={yshift=-3pt}]
                \addplot[2PL-BarStyle, forget plot] coordinates {(-0.5, 0.000607)};
                \addplot[KR-OCC-BarStyle, forget plot] coordinates {(-0.26, 0.000509)};
                \addplot[Memory-BarStyle] coordinates {(0.425, 0.00026) (1.7125, 0.000032) (3, 0.000075)};
                \addplot[Disk-BarStyle] coordinates {(0.425, 0.000203) (1.7125, 0.000052) (3, 0.000076)};
                \addplot[Disk-Cache-BarStyle] coordinates {(0.425, 0.000178) (1.7125, 0.000175) (3, 0.000064)};
                \addplot[Counter-Lazy-BarStyle] coordinates {(0.425, 0.227664) (1.7125, 0.180463) (3, 0.000257)};
                \addplot[FPSketch-BarStyle] coordinates {(0.425, 0.128828) (1.7125, 0.078013) (3, 0.000214)};
                \coordinate (bot) at (rel axis cs:1,0);% coordinate at bottom of the last plot

            \end{groupplot}
            \coordinate (c) at ($(top)!.5!(bot)$);
            \coordinate (cc1) at ($(top)!.25!(bot)$);
            \coordinate (cc2) at ($(top)!.75!(bot)$);

            \node[above] at (c |- current bounding box.north) {\ref{grouplegend}};
            \node[below] at (cc1 |- current bounding box.south) {(a) High-contended (zipf=0.99)};
            \node[above] at (cc2 |- current bounding box.south) {(b) Medium-contended (zipf=0.6)};
        \end{tikzpicture}
    } \caption[YCSB mixed-transaction workloads results on slow SSD]{YCSB
    mixed-transaction workloads results with 120 processing threads on slow SSD.
    80\% of transactions perform 2 reads and 2 writes, and 20\% of transactions
    perform 28 reads. Note that, in the medium-contended workload, several
    variants have abort rates very close to 0.}
    \label{fig:ycsb:mixed:slow_ssd}
\end{figure}


In \Cref{fig:ycsb:mixed:slow_ssd}, both \sketchname variants show strong
performance in mixed-transaction settings on slow SSD. In high-contention
scenarios, \sketchname offers up to 60\% higher goodput than the Disk variants.
It also outperforms 2PL and KR-OCC, with goodput increases of up to 24\% and
59\%, respectively.

Under medium contention, the goodputs of these systems are mainly limited by
slow disk access. Unlike our evaluation with NVMe SSDs described in
\Cref{sec:eval}, the concurrency control method has little effect on overall
goodput. However, since the Disk variants must fetch timestamps from the disk,
there is extra overhead, resulting in lower goodput. \sketchname avoids this
disk overhead and can improve goodput by up to 40\%.




\subsubsection{YCSB Long-Transaction Workloads}


\begin{figure}[!t]
    \centering
    \resizebox{.75\textwidth}{!}{%
        \begin{tikzpicture}
            \begin{groupplot}[group style={
                            group size=2 by 1,
                            horizontal sep=1.5cm,
                            /pgf/bar width=2.7pt
                        },
                    height = 3.6cm,
                    width = 4.5cm,
                    ybar= 2*\pgflinewidth,
                    xtick={-1, 0.425, 1.7125, 3.1},
                    xticklabels={2PL/OCC, STO, MVTO, TicToc},
                    ymajorgrids=true,
                    enlarge x limits = 0.225,
                    legend columns=1,
                    legend entries={{\ssmall Memory (Idealized)}, {\ssmall Disk}, {\ssmall Disk-Cache}, {\ssmall \fsketchname},  {\ssmall \psketchname}},
                    area legend,
                    legend to name=grouplegend,
                ]

                % graph [1,1] high
                \nextgroupplot[YCSBThroughputBarChartPlot, ylabel={Goodput (KTPS)}, ylabel style={yshift=-5pt}]
                \addplot[2PL-BarStyle, forget plot] coordinates {(-0.5, 1.132117)};
                \addplot[KR-OCC-BarStyle, forget plot] coordinates {(-0.26, 2.381903)};
                \addplot[Memory-BarStyle] coordinates {(0.425, 4.298397) (1.7125, 2.3857) (3, 2.398567)};
                \addplot[Disk-BarStyle] coordinates {(0.425, 2.726483) (1.7125, 0.006448) (3, 1.47637)};
                \addplot[Disk-Cache-BarStyle] coordinates {(0.425, 2.55686) (1.7125, 1.613997) (3, 1.495053)};
                \addplot[Counter-Lazy-BarStyle] coordinates {(0.425, 4.369143) (1.7125, 2.26709) (3, 2.393783)};
                \addplot[FPSketch-BarStyle] coordinates {(0.425, 4.84888) (1.7125, 1.949793) (3, 2.39416)};
                \coordinate (top) at (rel axis cs:0,1);% coordinate at top of the first plot

                % graph [1,2] abort rate
                \nextgroupplot[AbortRateBarChartPlot, ylabel={Abort rate}, ylabel style={yshift=-5pt}]
                \addplot[2PL-BarStyle, forget plot] coordinates {(-0.5, 0.780498)};
                \addplot[KR-OCC-BarStyle, forget plot] coordinates {(-0.26, 0.338777)}; 
                \addplot[Memory-BarStyle] coordinates {(0.425, 0.306208) (1.7125, 0.271251) (3, 0.250567)};
                \addplot[Disk-BarStyle] coordinates {(0.425, 0.311134) (1.7125, 0.268125) (3, 0.24539)};
                \addplot[Disk-Cache-BarStyle] coordinates {(0.425, 0.340524) (1.7125, 0.275897) (3, 0.249158)};
                \addplot[Counter-Lazy-BarStyle] coordinates {(0.425, 0.30175) (1.7125, 0.249261) (3, 0.25168)};
                \addplot[FPSketch-BarStyle] coordinates {(0.425, 0.287686) (1.7125, 0.261718) (3, 0.268675)};
                \coordinate (bot) at (rel axis cs:1,0);% coordinate at bottom of the last plot

            \end{groupplot}
            \coordinate (c) at ($(top)!.5!(bot)$);

            \node[right=4cm] at (c |- current bounding box.east) {\ref{grouplegend}};
        \end{tikzpicture}
    }
    \caption[YCSB long-transaction workloads results on slow SSD]{YCSB long-transaction workloads results with 120 processing threads on slow SSD. 5\% of transactions perform 1000 reads and the rest of transactions perform 8 writes and 8 reads. The distribution is Zipfian with 0.9.}
    \label{fig:ycsb:long:slow_ssd}
\end{figure}


\Cref{fig:ycsb:long:slow_ssd} presents the goodput results for the
long-transaction workload on slow SSD. The main pattern is similar to previous
results: the \sketchname variants achieve higher goodput than their Disk
versions and 2PL/KR-OCC. Because the disk is slow, long transactions run more
slowly and the differences between each method are smaller than with the faster
NVMe SSD shown in \Cref{fig:ycsb:long}. Still, although \sketchname is not as
effective as with NVMe SSD, it continues to deliver the best performance among
all tested protocols.




\subsubsection{TPC-C Workloads}


\begin{figure*}[!t]
    \centering
    \resizebox{\textwidth}{!}{%
        \begin{tikzpicture}
            \begin{groupplot}[group style={group size=4 by 2,
                            /pgf/bar width=2.7pt},
                    height = 3.6cm,
                    width = 4.5cm,
                    ybar= 2*\pgflinewidth,
                    xtick={-0.8625, 0.425, 1.7125, 3},
                    xticklabels={2PL/OCC, STO, MVTO, TicToc},
                    xticklabel style={font=\tiny},
                    ymajorgrids=true,
                    enlarge x limits = 0.2,
                    legend columns=-1,
                    legend entries={{\ssmall Memory (Idealized)}, {\ssmall Disk}, {\ssmall Disk-Cache}, {\ssmall \fsketchname},  {\ssmall \psketchname}},
                    area legend,
                    legend to name=grouplegend,
                ]

                % graph [1,1]
                \nextgroupplot[title={4 warehouses},YCSBThroughputBarChartPlot, ylabel={Goodput (KTPS)}]
                \addplot[2PL-BarStyle, forget plot] coordinates {(-0.5, 10.435867)};
                \addplot[KR-OCC-BarStyle, forget plot] coordinates {(-0.26, 6.705303)};
                \addplot[Memory-BarStyle] coordinates {(0.425, 15.243133) (1.7125, 10.439267) (3, 16.118167)};
                \addplot[Disk-BarStyle] coordinates {(0.425, 10.630633) (1.7125, 0.487886) (3, 12.915967)};
                \addplot[Disk-Cache-BarStyle] coordinates {(0.425, 10.555533) (1.7125, 6.000767) (3, 10.588833)};
                \addplot[Counter-Lazy-BarStyle] coordinates {(0.425, 15.047167) (1.7125, 7.931357) (3, 14.545967)};
                \addplot[FPSketch-BarStyle] coordinates {(0.425, 15.4038) (1.7125, 7.893797) (3, 15.6993)};
                \coordinate (top) at (rel axis cs:0,1);% coordinate at top of the first plot

                % graph [1,2]
                \nextgroupplot[title={8 warehouses}, YCSBThroughputBarChartPlot]
                \addplot[2PL-BarStyle, forget plot] coordinates {(-0.5, 10.954033)};
                \addplot[KR-OCC-BarStyle, forget plot] coordinates {(-0.26, 4.538743)};
                \addplot[Memory-BarStyle] coordinates {(0.425, 12.9659) (1.7125, 9.66877) (3, 13.214967)};
                \addplot[Disk-BarStyle] coordinates {(0.425, 10.404567) (1.7125, 0.370496) (3, 11.023467)};
                \addplot[Disk-Cache-BarStyle] coordinates {(0.425, 10.5734) (1.7125, 5.628667) (3, 10.133817)};
                \addplot[Counter-Lazy-BarStyle] coordinates {(0.425, 12.979) (1.7125, 6.524537) (3, 12.9708)};
                \addplot[FPSketch-BarStyle] coordinates {(0.425, 13.353233) (1.7125, 6.707457) (3, 13.025)};

                % graph [1,3]
                \nextgroupplot[title={16 warehouses},YCSBThroughputBarChartPlot]
                \addplot[2PL-BarStyle, forget plot] coordinates {(-0.5, 10.092567)};
                \addplot[KR-OCC-BarStyle, forget plot] coordinates {(-0.26, 3.257557)};
                \addplot[Memory-BarStyle] coordinates {(0.425, 9.537317) (1.7125, 7.649797) (3, 9.185967)};
                \addplot[Disk-BarStyle] coordinates {(0.425, 7.545187) (1.7125, 0.224447) (3, 7.702217)};
                \addplot[Disk-Cache-BarStyle] coordinates {(0.425, 8.662643) (1.7125, 4.814237) (3, 8.409643)};
                \addplot[Counter-Lazy-BarStyle] coordinates {(0.425, 10.459933) (1.7125, 5.09145) (3, 9.528323)};
                \addplot[FPSketch-BarStyle] coordinates {(0.425, 9.873883) (1.7125, 4.969443) (3, 9.350587)};


                % graph [1,4] 32wh
                \nextgroupplot[title={32 warehouses},YCSBThroughputBarChartPlot]
                \addplot[2PL-BarStyle, forget plot] coordinates {(-0.5, 8.402127)};
                \addplot[KR-OCC-BarStyle, forget plot] coordinates {(-0.26, 3.368457)};
                \addplot[Memory-BarStyle] coordinates {(0.425, 7.48262) (1.7125, 6.326153) (3, 7.225083)};
                \addplot[Disk-BarStyle] coordinates {(0.425, 5.781087) (1.7125, 0.077289) (3, 6.190753)};
                \addplot[Disk-Cache-BarStyle] coordinates {(0.425, 6.90269) (1.7125, 3.958337) (3, 6.694893)};
                \addplot[Counter-Lazy-BarStyle] coordinates {(0.425, 8.722823) (1.7125, 4.49392) (3, 7.51293)};
                \addplot[FPSketch-BarStyle] coordinates {(0.425, 7.79229) (1.7125, 4.189127) (3, 7.45213)};

                \coordinate (bot) at (rel axis cs:1,0);% coordinate at bottom of the last plot

                % graph [2,1] 4 wh
                \nextgroupplot[AbortRateBarChartPlot, ylabel={Abort rate}]
                \addplot[2PL-BarStyle, forget plot] coordinates {(-0.5, 0.329391)};
                \addplot[KR-OCC-BarStyle, forget plot] coordinates {(-0.26, 0.586844)};
                \addplot[Memory-BarStyle] coordinates {(0.425, 0.305108) (1.7125, 0.348413) (3, 0.351015)};
                \addplot[Disk-BarStyle] coordinates {(0.425, 0.301385) (1.7125, 0.356276) (3, 0.364606)};
                \addplot[Disk-Cache-BarStyle] coordinates {(0.425, 0.302807) (1.7125, 0.385476) (3, 0.366083)};
                \addplot[Counter-Lazy-BarStyle] coordinates {(0.425, 0.326195) (1.7125, 0.499632) (3, 0.372108)};
                \addplot[FPSketch-BarStyle] coordinates {(0.425, 0.321034) (1.7125, 0.398993) (3, 0.370333)};

                % graph [2,2] 8 wh
                \nextgroupplot[AbortRateBarChartPlot]
                \addplot[2PL-BarStyle, forget plot] coordinates {(-0.5, 0.267192)};
                \addplot[KR-OCC-BarStyle, forget plot] coordinates {(-0.26, 0.626045)};
                \addplot[Memory-BarStyle] coordinates {(0.425, 0.254501) (1.7125, 0.278394) (3, 0.329011)};
                \addplot[Disk-BarStyle] coordinates {(0.425, 0.260583) (1.7125, 0.289969) (3, 0.33647)};
                \addplot[Disk-Cache-BarStyle] coordinates {(0.425, 0.263124) (1.7125, 0.306613) (3, 0.32157)};
                \addplot[Counter-Lazy-BarStyle] coordinates {(0.425, 0.335892) (1.7125, 0.519921) (3, 0.359967)};
                \addplot[FPSketch-BarStyle] coordinates {(0.425, 0.298861) (1.7125, 0.32861) (3, 0.36367)};

                % graph [2,3] 16 wh
                \nextgroupplot[AbortRateBarChartPlot]
                \addplot[2PL-BarStyle, forget plot] coordinates {(-0.5, 0.180696)};
                \addplot[KR-OCC-BarStyle, forget plot] coordinates {(-0.26, 0.618218)};
                \addplot[Memory-BarStyle] coordinates {(0.425, 0.205467) (1.7125, 0.207115) (3, 0.298383)};
                \addplot[Disk-BarStyle] coordinates {(0.425, 0.197486) (1.7125, 0.204732) (3, 0.288648)};
                \addplot[Disk-Cache-BarStyle] coordinates {(0.425, 0.206902) (1.7125, 0.197685) (3, 0.279051)};
                \addplot[Counter-Lazy-BarStyle] coordinates {(0.425, 0.395279) (1.7125, 0.525739) (3, 0.361279)};
                \addplot[FPSketch-BarStyle] coordinates {(0.425, 0.334312) (1.7125, 0.252467) (3, 0.356596)};

                % graph [2,4] 32 wh
                \nextgroupplot[AbortRateBarChartPlot]
                \addplot[2PL-BarStyle, forget plot] coordinates {(-0.5, 0.13619)};  % return
                \addplot[KR-OCC-BarStyle, forget plot] coordinates {(-0.26, 0.498413)};
                \addplot[Memory-BarStyle] coordinates {(0.425, 0.065079) (1.7125, 0.05539) (3, 0.218316)};
                \addplot[Disk-BarStyle] coordinates {(0.425, 0.05402) (1.7125, 0.062805) (3, 0.198286)};
                \addplot[Disk-Cache-BarStyle] coordinates {(0.425, 0.056568) (1.7125, 0.042959) (3, 0.193242)};
                \addplot[Counter-Lazy-BarStyle] coordinates {(0.425, 0.496261) (1.7125, 0.568785) (3, 0.302585)};
                \addplot[FPSketch-BarStyle] coordinates {(0.425, 0.373169) (1.7125, 0.189238) (3, 0.300333)};

                \addplot[Disk-Cache-BarStyle] coordinates {(0.425, 0) (1.7125, 0) (3, 0)};

            \end{groupplot}
            \coordinate (c) at ($(top)!.5!(bot)$);

            \node[above] at (c |- current bounding box.north) {\ref{grouplegend}};
        \end{tikzpicture}
    }
    \caption[TPC-C results on slow SSD]{TPC-C results with 120 processing threads on slow SSD (More warehouses means less contention).}
    \label{fig:tpcc:slow_ssd}
\end{figure*}

Figure~\ref{fig:tpcc:slow_ssd} shows the goodput for the same TPC-C workloads
described in \Cref{sec:eval-setup}. In general, the results show that the
\sketchname variants achieve higher goodput than the Disk variants because they
remove the I/O overhead caused by accessing timestamps. When compared to 2PL,
\sketchname variants achieve better goodput in high-contention cases,
specifically with 4 and 8 warehouses. However, 2PL performs a bit better than
TicToc with \sketchname. This difference comes from their abort patterns: 2PL is
more likely to abort \texttt{Payment} transactions, which are shorter, while
TicToc more often aborts the longer \texttt{NewOrder} transactions. For example,
in the 32-warehouse scenario, 93\% of aborts in 2PL are from \texttt{Payment}
transactions, but in TicToc, 79\% of aborts are from \texttt{NewOrder}
transactions. Because TicToc must re-execute aborted transactions, this hurts
its goodput, especially in environments with slow disks. As a result,
TicToc-\psketchname is 0.89$\times$ as fast as 2PL in the 32-warehouse case, but
in the 4-warehouse scenario, it is 1.5$\times$ faster than 2PL.

Having examined SATA SSD performance, we now turn to HDDs to evaluate \sketchname
under even more challenging storage conditions. HDDs represent the slowest storage
type in our evaluation, with latencies measured in milliseconds rather than
microseconds. This allows us to observe how \sketchname's effectiveness scales
as storage becomes increasingly I/O-bound and to understand its behavior in truly
slow storage environments where disk access dominates all other factors.


\subsection{HDD}

Hard disk drives (HDDs) are still widely used today for storing very large
amounts of data because they are inexpensive and can hold a lot of information.
For these experiments, we used a different server since CloudLab does not
provide the same machine with an HDD. The system we used for our HDD evaluation
has a 36-core Intel(R) Xeon(R) Gold 6154 CPU running at 3.00GHz (with 2 threads
per core), 192GiB of DDR4 2666 MHz memory, and an 893GiB SCSI HDD. In these
tests, we used 60 threads because the version of SplinterDB we ran requires
keeping some CPU cores available.



\subsubsection{YCSB Small-Transaction Workloads}

\begin{figure}[!t]
    \centering
    \resizebox{\textwidth}{!}{%
        \begin{tikzpicture}
            \begin{groupplot}[group style={group size=4 by 2,
                            /pgf/bar width=2.7pt},
                    height = 3.6cm,
                    width = 4.5cm,
                    ybar= 2*\pgflinewidth,
                    xtick={-1, 0.425, 1.7125, 3.1},
                    xticklabels={2PL/OCC, STO, MVTO, TicToc},
                    ymajorgrids=true,
                    enlarge x limits = 0.2,
                    legend columns=-1,
                    legend entries={{\ssmall Memory (Idealized)}, {\ssmall Disk}, {\ssmall Disk-Cache}, {\ssmall \fsketchname},  {\ssmall \psketchname}},
                    area legend,
                    legend to name=grouplegend,
                ]

                % graph [1,1] high
                \nextgroupplot[title={Read-intensive},YCSBThroughputBarChartPlot, ylabel={Goodput (KTPS)}]
                \addplot[2PL-BarStyle, forget plot] coordinates {(-0.5, 8.04716)};
                \addplot[KR-OCC-BarStyle, forget plot] coordinates {(-0.26, 14.695467)};
                \addplot[Memory-BarStyle] coordinates {(0.425, 12.507633) (1.7125, 12.443733) (3, 15.512933)};
                \addplot[Disk-BarStyle] coordinates {(0.425, 11.482667) (1.7125, 0.032009) (3, 11.2091)};
                \addplot[Disk-Cache-BarStyle] coordinates {(0.425, 9.275193) (1.7125, 8.73535) (3, 10.287267)};
                \addplot[Counter-Lazy-BarStyle] coordinates {(0.425, 12.488167) (1.7125, 11.936367) (3, 13.065367)};
                \addplot[FPSketch-BarStyle] coordinates {(0.425, 12.669733) (1.7125, 11.999567) (3, 15.265833)};
                \coordinate (top) at (rel axis cs:0,1);% coordinate at top of the first plot

                % graph [1,3] high
                \nextgroupplot[title={Write-intensive}, YCSBThroughputBarChartPlot]
                \addplot[2PL-BarStyle, forget plot] coordinates {(-0.5, 4.608637)};
                \addplot[KR-OCC-BarStyle, forget plot] coordinates {(-0.26, 18.0974)};
                \addplot[Memory-BarStyle] coordinates {(0.425, 7.979627) (1.7125, 5.519357) (3, 35.047833)};
                \addplot[Disk-BarStyle] coordinates {(0.425, 4.551933) (1.7125, 0.022174) (3, 5.246687)};
                \addplot[Disk-Cache-BarStyle] coordinates {(0.425, 4.043277) (1.7125, 3.439533) (3, 4.70213)};
                \addplot[Counter-Lazy-BarStyle] coordinates {(0.425, 8.250597) (1.7125, 5.342587) (3, 31.205367)};
                \addplot[FPSketch-BarStyle] coordinates {(0.425, 7.749817) (1.7125, 5.59362) (3, 32.499733)};

                % graph [1,1] medium
                \nextgroupplot[title={Read-intensive},YCSBThroughputBarChartPlot]
                \addplot[2PL-BarStyle, forget plot] coordinates {(-0.5, 5.74492)};
                \addplot[KR-OCC-BarStyle, forget plot] coordinates {(-0.26, 5.701743)};
                \addplot[Memory-BarStyle] coordinates {(0.425, 5.495853) (1.7125, 5.41075) (3, 5.564913)};
                \addplot[Disk-BarStyle] coordinates {(0.425, 4.99792) (1.7125, 0.02941) (3, 4.655443)};
                \addplot[Disk-Cache-BarStyle] coordinates {(0.425, 4.54806) (1.7125, 4.207397) (3, 4.484957)};
                \addplot[Counter-Lazy-BarStyle] coordinates {(0.425, 5.40575) (1.7125, 5.208967) (3, 5.57541)};
                \addplot[FPSketch-BarStyle] coordinates {(0.425, 5.540247) (1.7125, 5.27708) (3, 5.5448)};


                % graph [1,3] medium
                \nextgroupplot[title={Write-intensive},YCSBThroughputBarChartPlot]
                \addplot[2PL-BarStyle, forget plot] coordinates {(-0.5, 10.272067)};
                \addplot[KR-OCC-BarStyle, forget plot] coordinates {(-0.26, 10.4202)};
                \addplot[Memory-BarStyle] coordinates {(0.425, 10.319067) (1.7125, 9.46724) (3, 10.3047)};
                \addplot[Disk-BarStyle] coordinates {(0.425, 4.80508) (1.7125, 0.029812) (3, 4.726683)};
                \addplot[Disk-Cache-BarStyle] coordinates {(0.425, 4.47477) (1.7125, 3.780167) (3, 4.576263)};
                \addplot[Counter-Lazy-BarStyle] coordinates {(0.425, 10.01148) (1.7125, 6.948757) (3, 10.13065)};
                \addplot[FPSketch-BarStyle] coordinates {(0.425, 10.33551) (1.7125, 7.09663) (3, 10.452133)};
                \coordinate (bot) at (rel axis cs:1,0);% coordinate at bottom of the last plot

                % graph [1,2] read intensive
                \nextgroupplot[AbortRateBarChartPlot, ylabel={Abort rate}]
                \addplot[2PL-BarStyle, forget plot] coordinates {(-0.5, 0.247383)};
                \addplot[KR-OCC-BarStyle, forget plot] coordinates {(-0.26, 0.301227)};
                \addplot[Memory-BarStyle] coordinates {(0.425, 0.222445) (1.7125, 0.206034) (3, 0.07278)};
                \addplot[Disk-BarStyle] coordinates {(0.425, 0.21628) (1.7125, 0.25627) (3, 0.074478)};
                \addplot[Disk-Cache-BarStyle] coordinates {(0.425, 0.23351) (1.7125, 0.228067) (3, 0.073814)};
                \addplot[Counter-Lazy-BarStyle] coordinates {(0.425, 0.228598) (1.7125, 0.218929) (3, 0.103984)};
                \addplot[FPSketch-BarStyle] coordinates {(0.425, 0.221732) (1.7125, 0.216595) (3, 0.075632)};

                % graph [1,4] write intensive
                \nextgroupplot[AbortRateBarChartPlot]
                \addplot[2PL-BarStyle, forget plot] coordinates {(-0.5, 0.465557)};
                \addplot[KR-OCC-BarStyle, forget plot] coordinates {(-0.26, 0.34286)};
                \addplot[Memory-BarStyle] coordinates {(0.425, 0.400389) (1.7125, 0.306523) (3, 0.197603)};
                \addplot[Disk-BarStyle] coordinates {(0.425, 0.476595) (1.7125, 0.454508) (3, 0.243955)};
                \addplot[Disk-Cache-BarStyle] coordinates {(0.425, 0.494111) (1.7125, 0.444139) (3, 0.258452)};
                \addplot[Counter-Lazy-BarStyle] coordinates {(0.425, 0.394383) (1.7125, 0.309213) (3, 0.21479)};
                \addplot[FPSketch-BarStyle] coordinates {(0.425, 0.399936) (1.7125, 0.315027) (3, 0.203892)};

                % graph [1,2]
                \nextgroupplot[AbortRateBarChartPlot]
                \addplot[2PL-BarStyle, forget plot] coordinates {(-0.5, 0.000222)};
                \addplot[KR-OCC-BarStyle, forget plot] coordinates {(-0.26, 0.00011)};
                \addplot[Memory-BarStyle] coordinates {(0.425, 0.000047) (1.7125, 0.000047) (3, 0.000017)};
                \addplot[Disk-BarStyle] coordinates {(0.425, 0.000039) (1.7125, 0.000046) (3, 0.000024)};
                \addplot[Disk-Cache-BarStyle] coordinates {(0.425, 0.000045) (1.7125, 0.000044) (3, 0.000016)};
                \addplot[Counter-Lazy-BarStyle] coordinates {(0.425, 0.328226) (1.7125, 0.330902) (3, 0.000046)};
                \addplot[FPSketch-BarStyle] coordinates {(0.425, 0.000279) (1.7125, 0.000117) (3, 0.000039)};

                % graph [1,4]
                \nextgroupplot[AbortRateBarChartPlot]
                \addplot[2PL-BarStyle, forget plot] coordinates {(-0.5, 0.001511)};
                \addplot[KR-OCC-BarStyle, forget plot] coordinates {(-0.26, 0.000493)};
                \addplot[Memory-BarStyle] coordinates {(0.425, 0.000238) (1.7125, 0.000219) (3, 0.000093)};
                \addplot[Disk-BarStyle] coordinates {(0.425, 0.000276) (1.7125, 0.000137) (3, 0.000174)};
                \addplot[Disk-Cache-BarStyle] coordinates {(0.425, 0.000238) (1.7125, 0.000242) (3, 0.00018)};
                \addplot[Counter-Lazy-BarStyle] coordinates {(0.425, 0.427524) (1.7125, 0.3174) (3, 0.00022)};
                \addplot[FPSketch-BarStyle] coordinates {(0.425, 0.016898) (1.7125, 0.003473) (3, 0.000212)};

            \end{groupplot}
            \coordinate (c) at ($(top)!.5!(bot)$);
            \coordinate (cc1) at ($(top)!.25!(bot)$);
            \coordinate (cc2) at ($(top)!.75!(bot)$);

            \node[above] at (c |- current bounding box.north) {\ref{grouplegend}};
            \node[below] at (cc1 |- current bounding box.south) {(a) High-contended YCSB (zipf=0.99)};
            \node[above] at (cc2 |- current bounding box.south) {(b) Medium-contended YCSB (zipf=0.6)};
        \end{tikzpicture}
    } \caption[YCSB small-transaction workloads results on HDD]{YCSB
    small-transaction workloads results with 60 processing threads on HDD. In
    write-intensive workloads, each transaction performs 8 reads and 8 writes.
    In read-intensive workloads, each transaction performs 15 reads and 1 write.
    Note that, in the medium-contended workload, several variants have abort
    rates very close to 0.}
    \label{fig:ycsb:slow_hdd}
\end{figure}


\Cref{fig:ycsb:slow_hdd} shows the results of running YCSB small-transaction
workloads with 60 processing threads on an HDD. In the high-contention,
read-intensive workload, KR-OCC reaches about 14 KTPS. Because disk I/O is a
major bottleneck with slow storage, KR-OCC performs better than other
timestamp-based concurrency control (CC) methods except for TicToc.
TicToc-\psketchname has the highest goodput in this workload, running 1.04 times
faster than KR-OCC. In addition, \psketchname increases TicToc's goodput by 36\%
compared to TicToc-Disk.

Similar to the results in \Cref{fig:ycsb:slow_ssd}, when the workload is
high-contention and write-intensive, \sketchname keeps TicToc's high
performance. TicToc-\psketchname achieves 1.8 times the speed of KR-OCC and
increases goodput by 519\% compared to TicToc-Disk.

For workloads with medium contention, all methods are mainly limited by I/O. The
Disk versions have extra I/O overhead to get timestamps. \sketchname removes
this overhead, increasing goodput by up to 19\% and 121\% for the read-intensive
and the write-intensive workloads, respectively.



\subsubsection{YCSB Mixed-Transaction Workloads}


\begin{figure}[!t]
    \centering
    \resizebox{\textwidth}{!}{%
        \begin{tikzpicture}
            \begin{groupplot}[group style={group size=4 by 1,
                            /pgf/bar width=2.7pt},
                    height = 3.6cm,
                    width = 4.5cm,
                    ybar= 2*\pgflinewidth,
                    xtick={-1, 0.425, 1.7125, 3.1},
                    xticklabels={2PL/OCC, STO, MVTO, TicToc},
                    ymajorgrids=true,
                    enlarge x limits = 0.225,
                    legend columns=-1,
                    legend entries={{\ssmall Memory (Idealized)}, {\ssmall Disk}, {\ssmall Disk-Cache}, {\ssmall \fsketchname},  {\ssmall \psketchname}},
                    area legend,
                    legend to name=grouplegend,
                ]

                % graph [1,1] high
                \nextgroupplot[YCSBThroughputBarChartPlot, ylabel={Goodput (KTPS)}, ylabel style={yshift=-5pt}]
                \addplot[2PL-BarStyle, forget plot] coordinates {(-0.5, 14.398367)};
                \addplot[KR-OCC-BarStyle, forget plot] coordinates {(-0.26, 22.5943)};
                \addplot[Memory-BarStyle] coordinates {(0.425, 18.620767) (1.7125, 20.2924) (3, 21.2467)};
                \addplot[Disk-BarStyle] coordinates {(0.425, 14.2852) (1.7125, 0.057283) (3, 13.813567)};
                \addplot[Disk-Cache-BarStyle] coordinates {(0.425, 11.8618) (1.7125, 10.774433) (3, 12.974133)};
                \addplot[Counter-Lazy-BarStyle] coordinates {(0.425, 13.572667) (1.7125, 11.1527) (3, 18.250733)};
                \addplot[FPSketch-BarStyle] coordinates {(0.425, 18.253) (1.7125, 17.012867) (3, 21.638033)};
                \coordinate (top) at (rel axis cs:0,1);% coordinate at top of the first plot

                % graph [2,1] abort rate
                \nextgroupplot[AbortRateBarChartPlot, ylabel={Abort rate}, ylabel style={yshift=-3pt}]
                \addplot[2PL-BarStyle, forget plot] coordinates {(-0.5, 0.120156)};
                \addplot[KR-OCC-BarStyle, forget plot] coordinates {(-0.26, 0.153397)};
                \addplot[Memory-BarStyle] coordinates {(0.425, 0.081903) (1.7125, 0.025545) (3, 0.034029)};
                \addplot[Disk-BarStyle] coordinates {(0.425, 0.083356) (1.7125, 0.034113) (3, 0.032248)};
                \addplot[Disk-Cache-BarStyle] coordinates {(0.425, 0.085848) (1.7125, 0.031777) (3, 0.031905)};
                \addplot[Counter-Lazy-BarStyle] coordinates {(0.425, 0.131356) (1.7125, 0.088476) (3, 0.05094)};
                \addplot[FPSketch-BarStyle] coordinates {(0.425, 0.086357) (1.7125, 0.07476) (3, 0.034597)};

                % graph [3,1] medium
                \nextgroupplot[YCSBThroughputBarChartPlot, ylabel={Goodput (KTPS)}, ylabel style={yshift=-8pt}]
                \addplot[2PL-BarStyle, forget plot] coordinates {(-0.5, 11.759533)};
                \addplot[KR-OCC-BarStyle, forget plot] coordinates {(-0.26, 11.808467)};
                \addplot[Memory-BarStyle] coordinates {(0.425, 11.777167) (1.7125, 11.1943) (3, 11.590367)};
                \addplot[Disk-BarStyle] coordinates {(0.425, 9.137557) (1.7125, 0.053173) (3, 8.6221)};
                \addplot[Disk-Cache-BarStyle] coordinates {(0.425, 8.24723) (1.7125, 7.554443) (3, 8.227663)};
                \addplot[Counter-Lazy-BarStyle] coordinates {(0.425, 8.86626) (1.7125, 7.247293) (3, 11.721167)};
                \addplot[FPSketch-BarStyle] coordinates {(0.425, 11.8456) (1.7125, 10.549667) (3, 11.9516)};

                % graph [4,1] medium abort rate
                \nextgroupplot[AbortRateBarChartPlot, ylabel={Abort rate}, ylabel style={yshift=-3pt}]
                \addplot[2PL-BarStyle, forget plot] coordinates {(-0.5, 0.000225)};
                \addplot[KR-OCC-BarStyle, forget plot] coordinates {(-0.26, 0.000255)};
                \addplot[Memory-BarStyle] coordinates {(0.425, 0.000116) (1.7125, 0.000018) (3, 0.000039)};
                \addplot[Disk-BarStyle] coordinates {(0.425, 0.000103) (1.7125, 0.000052) (3, 0.000041)};
                \addplot[Disk-Cache-BarStyle] coordinates {(0.425, 0.000092) (1.7125, 0.000077) (3, 0.000034)};
                \addplot[Counter-Lazy-BarStyle] coordinates {(0.425, 0.180175) (1.7125, 0.132656) (3, 0.000122)};
                \addplot[FPSketch-BarStyle] coordinates {(0.425, 0.061843) (1.7125, 0.029213) (3, 0.00009)};
                \coordinate (bot) at (rel axis cs:1,0);% coordinate at bottom of the last plot

            \end{groupplot}
            \coordinate (c) at ($(top)!.5!(bot)$);
            \coordinate (cc1) at ($(top)!.25!(bot)$);
            \coordinate (cc2) at ($(top)!.75!(bot)$);

            \node[above] at (c |- current bounding box.north) {\ref{grouplegend}};
            \node[below] at (cc1 |- current bounding box.south) {(a) High-contended (zipf=0.99)};
            \node[above] at (cc2 |- current bounding box.south) {(b) Medium-contended (zipf=0.6)};
        \end{tikzpicture}
    } \caption[YCSB mixed-transaction workloads results on HDD]{YCSB
    mixed-transaction workloads results with 60 processing threads on HDD. 80\%
    of transactions perform 2 reads and 2 writes, and 20\% of transactions
    perform 28 reads. Note that, in the medium-contended workload, several
    variants have abort rates very close to 0.}
    \label{fig:ycsb:mixed:hdd}
\end{figure}

\Cref{fig:ycsb:mixed:hdd} shows the results of running YCSB mixed-transaction
workloads with 60 processing threads on an HDD. The results are similar to the
read-intensive workload results on HDD in \Cref{fig:ycsb:slow_hdd} because 20\%
of transactions perform 28 reads. TicToc-\psketchname is 0.96$\times$ slower
than KR-OCC but it increases goodput by 56.6\% compared to TicToc-Disk.



\subsubsection{YCSB Long-Transaction Workload}



\begin{figure}[!t]
    \centering
    \resizebox{.75\textwidth}{!}{%
        \begin{tikzpicture}
            \begin{groupplot}[group style={
                            group size=2 by 1,
                            horizontal sep=1.5cm,
                            /pgf/bar width=2.7pt
                        },
                    height = 3.6cm,
                    width = 4.5cm,
                    ybar= 2*\pgflinewidth,
                    xtick={-1, 0.425, 1.7125, 3.1},
                    xticklabels={2PL/OCC, STO, MVTO, TicToc},
                    ymajorgrids=true,
                    enlarge x limits = 0.225,
                    legend columns=1,
                    legend entries={{\ssmall Memory (Idealized)}, {\ssmall Disk}, {\ssmall Disk-Cache}, {\ssmall \fsketchname},  {\ssmall \psketchname}},
                    area legend,
                    legend to name=grouplegend,
                ]

                % graph [1,1] high
                \nextgroupplot[YCSBThroughputBarChartPlot, ylabel={Goodput (KTPS)}, ylabel style={yshift=-5pt}]
                \addplot[2PL-BarStyle, forget plot] coordinates {(-0.5, 0.978672)};
                \addplot[KR-OCC-BarStyle, forget plot] coordinates {(-0.26, 2.292817)};
                \addplot[Memory-BarStyle] coordinates {(0.425, 2.254873) (1.7125, 2.272397) (3, 2.263723)};
                \addplot[Disk-BarStyle] coordinates {(0.425, 1.441723) (1.7125, 0.006307) (3, 1.035913)};
                \addplot[Disk-Cache-BarStyle] coordinates {(0.425, 1.399493) (1.7125, 1.361513) (3, 1.10648)};
                \addplot[Counter-Lazy-BarStyle] coordinates {(0.425, 2.25934) (1.7125, 1.798883) (3, 2.244893)};
                \addplot[FPSketch-BarStyle] coordinates {(0.425, 2.364233) (1.7125, 1.121917) (3, 2.22647)};
                \coordinate (top) at (rel axis cs:0,1);% coordinate at top of the first plot

                % graph [1,2] abort rate
                \nextgroupplot[AbortRateBarChartPlot, ylabel={Abort rate}, ylabel style={yshift=-5pt}]
                \addplot[2PL-BarStyle, forget plot] coordinates {(-0.5, 0.684035)};
                \addplot[KR-OCC-BarStyle, forget plot] coordinates {(-0.26, 0.282721)}; % double check
                \addplot[Memory-BarStyle] coordinates {(0.425, 0.259254) (1.7125, 0.128753) (3, 0.224552)};
                \addplot[Disk-BarStyle] coordinates {(0.425, 0.264165) (1.7125, 0.365076) (3, 0.204642)}; % double check mvcc
                \addplot[Disk-Cache-BarStyle] coordinates {(0.425, 0.270411) (1.7125, 0.182728) (3, 0.199765)}; % rerun
                \addplot[Counter-Lazy-BarStyle] coordinates {(0.425, 0.258943) (1.7125, 0.189777) (3, 0.222702)};
                \addplot[FPSketch-BarStyle] coordinates {(0.425, 0.259826) (1.7125, 0.234995) (3, 0.227486)};
                \coordinate (bot) at (rel axis cs:1,0);% coordinate at bottom of the last plot

            \end{groupplot}
            \coordinate (c) at ($(top)!.5!(bot)$);

            \node[right=4cm] at (c |- current bounding box.east) {\ref{grouplegend}};
        \end{tikzpicture}
    } \caption[YCSB long-transaction workloads results on HDD]{YCSB
    long-transaction workloads results with 60 processing threads on HDD. 5\% of
    transactions perform 1000 reads and the rest of transactions perform 8
    writes and 8 reads. The distribution is Zipfian with 0.9.}
    \label{fig:ycsb:long:hdd}
\end{figure}

As the slow disk is a major bottleneck in read-intensive workloads, all methods
except 2PL show similar goodput, while 2PL causes higher abort rates. \sketchname
variants show their effectiveness in removing the need to access timestamps on
disk, resulting in higher goodput. However, MVTO with \sketchname incurs extra
write operations as described in \Cref{sec:fpsketch:mvto}, resulting in lower
goodput.


\subsubsection{TPC-C Workloads}

\begin{figure*}[!t]
    \centering
    \resizebox{\textwidth}{!}{%
        \begin{tikzpicture}
            \begin{groupplot}[group style={group size=4 by 2,
                            /pgf/bar width=2.7pt},
                    height = 3.6cm,
                    width = 4.5cm,
                    ybar= 2*\pgflinewidth,
                    xtick={-0.8625, 0.425, 1.7125, 3},
                    xticklabels={2PL/OCC, STO, MVTO, TicToc},
                    xticklabel style={font=\tiny},
                    ymajorgrids=true,
                    enlarge x limits = 0.2,
                    legend columns=-1,
                    legend entries={{\ssmall Memory (Idealized)}, {\ssmall Disk}, {\ssmall Disk-Cache}, {\ssmall \fsketchname},  {\ssmall \psketchname}},
                    area legend,
                    legend to name=grouplegend,
                ]

                % graph [1,1]
                \nextgroupplot[title={4 warehouses},YCSBThroughputBarChartPlot, ylabel={Goodput (KTPS)}]
                \addplot[2PL-BarStyle, forget plot] coordinates {(-0.5, 19.115367)};
                \addplot[KR-OCC-BarStyle, forget plot] coordinates {(-0.26, 18.856033)};
                \addplot[Memory-BarStyle] coordinates {(0.425, 23.608167) (1.7125, 14.4289) (3, 22.799833)};
                \addplot[Disk-BarStyle] coordinates {(0.425, 16.973967) (1.7125, 1.393113) (3, 18.8406)};
                \addplot[Disk-Cache-BarStyle] coordinates {(0.425, 16.312733) (1.7125, 9.138087) (3, 12.860733)};
                \addplot[Counter-Lazy-BarStyle] coordinates {(0.425, 22.9111) (1.7125, 12.039267) (3, 20.9325)};
                \addplot[FPSketch-BarStyle] coordinates {(0.425, 23.628533) (1.7125, 12.440333) (3, 22.200533)};
                \coordinate (top) at (rel axis cs:0,1);% coordinate at top of the first plot

                % graph [1,2]
                \nextgroupplot[title={8 warehouses}, YCSBThroughputBarChartPlot]
                \addplot[2PL-BarStyle, forget plot] coordinates {(-0.5, 18.381033)};
                \addplot[KR-OCC-BarStyle, forget plot] coordinates {(-0.26, 16.4903)};
                \addplot[Memory-BarStyle] coordinates {(0.425, 19.961433) (1.7125, 13.819533) (3, 19.699833)};
                \addplot[Disk-BarStyle] coordinates {(0.425, 15.508133) (1.7125, 1.069157) (3, 16.799167)};
                \addplot[Disk-Cache-BarStyle] coordinates {(0.425, 14.151633) (1.7125, 7.96076) (3, 12.205833)};
                \addplot[Counter-Lazy-BarStyle] coordinates {(0.425, 18.863267) (1.7125, 10.526333) (3, 19.065667)};
                \addplot[FPSketch-BarStyle] coordinates {(0.425, 20.064667) (1.7125, 10.490667) (3, 19.3602)};

                % graph [1,3]
                \nextgroupplot[title={16 warehouses},YCSBThroughputBarChartPlot]
                \addplot[2PL-BarStyle, forget plot] coordinates {(-0.5, 14.7968)};
                \addplot[KR-OCC-BarStyle, forget plot] coordinates {(-0.26, 12.914467)};
                \addplot[Memory-BarStyle] coordinates {(0.425, 14.465267) (1.7125, 10.981267) (3, 14.941067)};
                \addplot[Disk-BarStyle] coordinates {(0.425, 11.4572) (1.7125, 0.588741) (3, 12.457133)};
                \addplot[Disk-Cache-BarStyle] coordinates {(0.425, 10.463967) (1.7125, 6.40315) (3, 9.55108)};
                \addplot[Counter-Lazy-BarStyle] coordinates {(0.425, 14.398667) (1.7125, 8.379683) (3, 15.104167)};
                \addplot[FPSketch-BarStyle] coordinates {(0.425, 15.039833) (1.7125, 7.637783) (3, 15.2448)};


                % graph [1,4] 32wh
                \nextgroupplot[title={32 warehouses},YCSBThroughputBarChartPlot]
                \addplot[2PL-BarStyle, forget plot] coordinates {(-0.5, 11.7632)};
                \addplot[KR-OCC-BarStyle, forget plot] coordinates {(-0.26, 10.5966)};
                \addplot[Memory-BarStyle] coordinates {(0.425, 10.126013) (1.7125, 8.492203) (3, 10.425733)};
                \addplot[Disk-BarStyle] coordinates {(0.425, 8.154903) (1.7125, 0.066267) (3, 8.868137)};
                \addplot[Disk-Cache-BarStyle] coordinates {(0.425, 8.18903) (1.7125, 5.0047) (3, 7.765507)};
                \addplot[Counter-Lazy-BarStyle] coordinates {(0.425, 10.880967) (1.7125, 6.456127) (3, 10.9942)};
                \addplot[FPSketch-BarStyle] coordinates {(0.425, 11.699533) (1.7125, 5.712097) (3, 10.9556)};

                \coordinate (bot) at (rel axis cs:1,0);% coordinate at bottom of the last plot

                % graph [2,1] 4 wh
                \nextgroupplot[AbortRateBarChartPlot, ylabel={Abort rate}]
                \addplot[2PL-BarStyle, forget plot] coordinates {(-0.5, 0.135487)};
                \addplot[KR-OCC-BarStyle, forget plot] coordinates {(-0.26, 0.327479)};
                \addplot[Memory-BarStyle] coordinates {(0.425, 0.16263) (1.7125, 0.190511) (3, 0.181326)};
                \addplot[Disk-BarStyle] coordinates {(0.425, 0.168929) (1.7125, 0.164452) (3, 0.201198)};
                \addplot[Disk-Cache-BarStyle] coordinates {(0.425, 0.173134) (1.7125, 0.164581) (3, 0.226057)};
                \addplot[Counter-Lazy-BarStyle] coordinates {(0.425, 0.193154) (1.7125, 0.318299) (3, 0.188983)};
                \addplot[FPSketch-BarStyle] coordinates {(0.425, 0.170551) (1.7125, 0.205813) (3, 0.189242)};

                % graph [2,2] 8 wh
                \nextgroupplot[AbortRateBarChartPlot]
                \addplot[2PL-BarStyle, forget plot] coordinates {(-0.5, 0.102231)};
                \addplot[KR-OCC-BarStyle, forget plot] coordinates {(-0.26, 0.352503)};
                \addplot[Memory-BarStyle] coordinates {(0.425, 0.116895) (1.7125, 0.130499) (3, 0.165533)};
                \addplot[Disk-BarStyle] coordinates {(0.425, 0.13252) (1.7125, 0.11766) (3, 0.176234)};
                \addplot[Disk-Cache-BarStyle] coordinates {(0.425, 0.134825) (1.7125, 0.140453) (3, 0.183218)};
                \addplot[Counter-Lazy-BarStyle] coordinates {(0.425, 0.192119) (1.7125, 0.345666) (3, 0.177427)};
                \addplot[FPSketch-BarStyle] coordinates {(0.425, 0.144327) (1.7125, 0.138727) (3, 0.178054)};

                % graph [2,3] 16 wh
                \nextgroupplot[AbortRateBarChartPlot]
                \addplot[2PL-BarStyle, forget plot] coordinates {(-0.5, 0.064185)};
                \addplot[KR-OCC-BarStyle, forget plot] coordinates {(-0.26, 0.318935)};
                \addplot[Memory-BarStyle] coordinates {(0.425, 0.063774) (1.7125, 0.056843) (3, 0.15089)};
                \addplot[Disk-BarStyle] coordinates {(0.425, 0.062464) (1.7125, 0.075799) (3, 0.14947)};
                \addplot[Disk-Cache-BarStyle] coordinates {(0.425, 0.056327) (1.7125, 0.039754) (3, 0.147102)};
                \addplot[Counter-Lazy-BarStyle] coordinates {(0.425, 0.243272) (1.7125, 0.398401) (3, 0.178241)};
                \addplot[FPSketch-BarStyle] coordinates {(0.425, 0.17467) (1.7125, 0.121197) (3, 0.178254)};

                % graph [2,4] 32 wh
                \nextgroupplot[AbortRateBarChartPlot]
                \addplot[2PL-BarStyle, forget plot] coordinates {(-0.5, 0.036227)};
                \addplot[KR-OCC-BarStyle, forget plot] coordinates {(-0.26, 0.195207)};
                \addplot[Memory-BarStyle] coordinates {(0.425, 0.008601) (1.7125, 0.006069) (3, 0.086278)};
                \addplot[Disk-BarStyle] coordinates {(0.425, 0.007884) (1.7125, 0.004012) (3, 0.082685)};
                \addplot[Disk-Cache-BarStyle] coordinates {(0.425, 0.004574) (1.7125, 0.002209) (3, 0.081135)};
                \addplot[Counter-Lazy-BarStyle] coordinates {(0.425, 0.28479) (1.7125, 0.445157) (3, 0.12057)};
                \addplot[FPSketch-BarStyle] coordinates {(0.425, 0.20554) (1.7125, 0.116253) (3, 0.11906)};

                \addplot[Disk-Cache-BarStyle] coordinates {(0.425, 0) (1.7125, 0) (3, 0)};

            \end{groupplot}
            \coordinate (c) at ($(top)!.5!(bot)$);

            \node[above] at (c |- current bounding box.north) {\ref{grouplegend}};
        \end{tikzpicture}
    } \caption[TPC-C results on HDD]{TPC-C results with 60 processing threads
    on HDD (More warehouses means less contention).}
    \label{fig:tpcc:hdd}
\end{figure*}

Figure~\ref{fig:tpcc:hdd} presents the goodput results for the TPC-C workloads
described in \Cref{sec:eval-setup}, but now running on an HDD. As seen with the
YCSB experiments on HDD, \sketchname performs very well in high-contention
situations with a small number of warehouses. When contention is lower, however,
the performance of all concurrency control methods becomes similar because the
speed of the slow hard drive limits throughput. With 4 warehouses,
STO-\psketchname and TicToc-\psketchname achieve up to 1.25$\times$ and
1.18$\times$ better goodput than 2PL/KR-OCC, respectively.



\section{Fast Storage}\label{sec:fast-storage}

Having demonstrated \sketchname's effectiveness on slow storage, we now evaluate
its performance on fast storage media, such as CXL-based flash with DRAM-like 
latency. This evaluation reveals a fundamental shift in performance 
characteristics: as storage becomes faster, the system transitions from being 
I/O-bound to CPU-bound, making the overhead of the \sketchname data structure 
itself more visible relative to storage access costs. These results provide 
insights into how \sketchname variants scale as storage technologies continue to 
evolve toward lower latencies.

\subsection{CXL Emulation Methodology}

To evaluate \sketchname's effectiveness on next-generation storage technologies,
we emulate CXL-based flash storage using block ramdisk. This subsection details
our emulation methodology, the assumptions we make, and how our results relate to
real CXL hardware.

\subsubsection{Emulation Setup}

We use block ramdisk on the same CloudLab machine described in \Cref{sec:eval-setup}
to emulate CXL-based flash with DRAM-like latency. Block ramdisk provides storage
access through the standard block device interface while storing data in RAM,
effectively creating a software-based storage layer that mimics the low latency
characteristics of memory-like storage technologies. This approach allows us to
evaluate \sketchname's behavior when storage access times are dramatically reduced
without requiring specialized CXL hardware that may not be readily available in
experimental environments.

The emulated storage provides latencies approximately 100$\times$ faster than SATA
SSD and 10$\times$ faster than NVMe SSD, closely approximating the characteristics
of next-generation CXL-based storage. Specifically, while NVMe SSDs achieve
latencies in the 20--100 microsecond range, CXL-based storage targets latencies in
the single-digit microsecond range (1--10 microseconds). Our block ramdisk
emulation achieves latencies on the order of hundreds of nanoseconds, providing a
reasonable approximation of CXL performance characteristics for the purpose of
understanding how \sketchname scales as storage becomes faster.

\subsubsection{Assumptions and Limitations}

Our emulation makes several important assumptions that should be considered when
interpreting results:

\begin{itemize}
    \item \textbf{Interface semantics}: Block ramdisk uses the standard block
    device interface, whereas real CXL storage may support byte-addressable access
    through the CXL.mem protocol. This difference may affect the relative
    performance of different access patterns, though the fundamental latency
    characteristics we seek to emulate are preserved.
    
    \item \textbf{Bandwidth characteristics}: Real CXL storage provides high
    bandwidth (up to 16 GB/s on PCIe 5.0 x4), whereas our emulation is limited by
    system RAM bandwidth. However, since our evaluation focuses on latency-bound
    operations (random I/O for timestamp access), bandwidth differences have minimal
    impact on our findings.
    
    \item \textbf{Remote memory effects}: Real CXL storage may exhibit NUMA-like
    behavior with remote memory access characteristics. Our emulation does not
    capture these effects, though \sketchname's effectiveness in avoiding remote
    access overhead remains valid.
    
    \item \textbf{CPU overhead}: Block ramdisk still incurs some software overhead
    from the block device stack, whereas real CXL storage with direct memory access
    may have different overhead characteristics. However, this overhead is minimal
    compared to actual storage latency and does not materially affect our
    conclusions.
\end{itemize}

\subsubsection{Experimental Configuration}

In these fast storage experiments, we evaluate only STO and TicToc protocols.
MVTO runs out of memory on the machine with 256GB of memory because its write
operations keep creating new versions of keys, making it impractical for
evaluation on fast storage where write rates are extremely high. We run 120
processing threads for all experiments, matching the configuration used in our
NVMe SSD evaluations. All other experimental parameters (workloads, database size,
cache configurations) remain consistent with the evaluations described in
\Cref{sec:eval-setup} to ensure fair comparison across storage types.

The following sections present results showing that \sketchname variants
significantly outperform 2PL and KR-OCC on fast storage, achieving dramatic
improvements in high-contention scenarios while maintaining effectiveness across
all workload types. However, the relative overhead of \sketchname operations
becomes more pronounced, revealing areas for future optimization.

\subsubsection{YCSB Small-Transaction Workloads}


\begin{figure}[!t]
    \centering
    \resizebox{\textwidth}{!}{%
        \begin{tikzpicture}
            \begin{groupplot}[group style={group size=4 by 2,
                            /pgf/bar width=2.7pt},
                    height = 3.6cm,
                    width = 4.5cm,
                    ybar= 2*\pgflinewidth,
                    xtick={-1, 0.425, 3.1},
                    xticklabels={2PL/OCC, STO, TicToc},
                    ymajorgrids=true,
                    enlarge x limits = 0.2,
                    legend columns=-1,
                    legend entries={{\ssmall Memory (Idealized)}, {\ssmall Disk}, {\ssmall Disk-Cache}, {\ssmall \fsketchname},  {\ssmall \psketchname}},
                    area legend,
                    legend to name=grouplegend,
                ]

                % graph [1,1] high
                \nextgroupplot[title={Read-intensive},YCSBThroughputBarChartPlot, ylabel={Goodput (KTPS)}]
                \addplot[2PL-BarStyle, forget plot] coordinates {(-0.5, 58.2401)};
                \addplot[KR-OCC-BarStyle, forget plot] coordinates {(-0.26, 9.641523)};
                \addplot[Memory-BarStyle] coordinates {(0.425, 100.589433) (3, 342.33)};
                \addplot[Disk-BarStyle] coordinates {(0.425, 32.581733) (3, 53.617733)};
                \addplot[Disk-Cache-BarStyle] coordinates {(0.425, 34.518667) (3, 60.1803)};
                \addplot[Counter-Lazy-BarStyle] coordinates {(0.425, 52.053633) (3, 105.591)};
                \addplot[FPSketch-BarStyle] coordinates {(0.425, 78.006133) (3, 206.962)};
                \coordinate (top) at (rel axis cs:0,1);% coordinate at top of the first plot

                % graph [1,3] high
                \nextgroupplot[title={Write-intensive}, YCSBThroughputBarChartPlot]
                \addplot[2PL-BarStyle, forget plot] coordinates {(-0.5, 19.146733)};
                \addplot[KR-OCC-BarStyle, forget plot] coordinates {(-0.26, 12.871567)};
                \addplot[Memory-BarStyle] coordinates {(0.425, 20.868633) (3, 48.751833)};
                \addplot[Disk-BarStyle] coordinates {(0.425, 11.414233) (3, 13.5591)};
                \addplot[Disk-Cache-BarStyle] coordinates {(0.425, 10.320467) (3, 12.3741)};
                \addplot[Counter-Lazy-BarStyle] coordinates {(0.425, 17.8436) (3, 30.488667)};
                \addplot[FPSketch-BarStyle] coordinates {(0.425, 19.657167) (3, 40.6375)};

                % graph [1,1] medium
                \nextgroupplot[title={Read-intensive},YCSBThroughputBarChartPlot]
                \addplot[2PL-BarStyle, forget plot] coordinates {(-0.5, 272.525667)};
                \addplot[KR-OCC-BarStyle, forget plot] coordinates {(-0.26, 5.632817)};
                \addplot[Memory-BarStyle] coordinates {(0.425, 304.898) (3, 305.503333)};
                \addplot[Disk-BarStyle] coordinates {(0.425, 189.604) (3, 113.674333)};
                \addplot[Disk-Cache-BarStyle] coordinates {(0.425, 71.498267) (3, 71.392467)};
                \addplot[Counter-Lazy-BarStyle] coordinates {(0.425, 149.862667) (3, 151.15)};
                \addplot[FPSketch-BarStyle] coordinates {(0.425, 235.684667) (3, 235.993333)};


                % graph [1,3] medium
                \nextgroupplot[title={Write-intensive},YCSBThroughputBarChartPlot]
                \addplot[2PL-BarStyle, forget plot] coordinates {(-0.5, 299.059333)};
                \addplot[KR-OCC-BarStyle, forget plot] coordinates {(-0.26, 5.085443)};
                \addplot[Memory-BarStyle] coordinates {(0.425, 293.822) (3, 293.970333)};
                \addplot[Disk-BarStyle] coordinates {(0.425, 154.005) (3, 146.303)};
                \addplot[Disk-Cache-BarStyle] coordinates {(0.425, 70.924433) (3, 71.719567)};
                \addplot[Counter-Lazy-BarStyle] coordinates {(0.425, 156.197) (3, 161.700667)};
                \addplot[FPSketch-BarStyle] coordinates {(0.425, 255.064) (3, 255.141333)};
                \coordinate (bot) at (rel axis cs:1,0);% coordinate at bottom of the last plot

                % graph [1,2] read intensive
                \nextgroupplot[AbortRateBarChartPlot, ylabel={Abort rate}]
                \addplot[2PL-BarStyle, forget plot] coordinates {(-0.5, 0.260427)};
                \addplot[KR-OCC-BarStyle, forget plot] coordinates {(-0.26, 0.415827)};
                \addplot[Memory-BarStyle] coordinates {(0.425, 0.314407) (3, 0.130448)};
                \addplot[Disk-BarStyle] coordinates {(0.425, 0.300011) (3, 0.136726)};
                \addplot[Disk-Cache-BarStyle] coordinates {(0.425, 0.296994) (3, 0.130361)};
                \addplot[Counter-Lazy-BarStyle] coordinates {(0.425, 0.386568) (3, 0.256559)};
                \addplot[FPSketch-BarStyle] coordinates {(0.425, 0.330712) (3, 0.149822)};

                % graph [1,4] write intensive
                \nextgroupplot[AbortRateBarChartPlot]
                \addplot[2PL-BarStyle, forget plot] coordinates {(-0.5, 0.513741)};
                \addplot[KR-OCC-BarStyle, forget plot] coordinates {(-0.26, 0.493239)};
                \addplot[Memory-BarStyle] coordinates {(0.425, 0.517988) (3, 0.28678)};
                \addplot[Disk-BarStyle] coordinates {(0.425, 0.516791) (3, 0.27266)};
                \addplot[Disk-Cache-BarStyle] coordinates {(0.425, 0.535364) (3, 0.292757)};
                \addplot[Counter-Lazy-BarStyle] coordinates {(0.425, 0.538097) (3, 0.337119)};
                \addplot[FPSketch-BarStyle] coordinates {(0.425, 0.517645) (3, 0.297033)};

                % graph [1,2]
                \nextgroupplot[AbortRateBarChartPlot]
                \addplot[2PL-BarStyle, forget plot] coordinates {(-0.5, 0.000413)};
                \addplot[KR-OCC-BarStyle, forget plot] coordinates {(-0.26, 0.000167)};
                \addplot[Memory-BarStyle] coordinates {(0.425, 0.000085) (3, 0.000031)};
                \addplot[Disk-BarStyle] coordinates {(0.425, 0.000081) (3, 0.000045)};
                \addplot[Disk-Cache-BarStyle] coordinates {(0.425, 0.000098) (3, 0.000032)};
                \addplot[Counter-Lazy-BarStyle] coordinates {(0.425, 0.001611) (3, 0.000105)};
                \addplot[FPSketch-BarStyle] coordinates {(0.425, 0.000211) (3, 0.00008)};

                % graph [1,4]
                \nextgroupplot[AbortRateBarChartPlot]
                \addplot[2PL-BarStyle, forget plot] coordinates {(-0.5, 0.002108)};
                \addplot[KR-OCC-BarStyle, forget plot] coordinates {(-0.26, 0.001008)};
                \addplot[Memory-BarStyle] coordinates {(0.425, 0.000498) (3, 0.000191)};
                \addplot[Disk-BarStyle] coordinates {(0.425, 0.000456) (3, 0.000237)};
                \addplot[Disk-Cache-BarStyle] coordinates {(0.425, 0.000605) (3, 0.000398)};
                \addplot[Counter-Lazy-BarStyle] coordinates {(0.425, 0.007669) (3, 0.000779)};
                \addplot[FPSketch-BarStyle] coordinates {(0.425, 0.003707) (3, 0.000566)};

            \end{groupplot}
            \coordinate (c) at ($(top)!.5!(bot)$);
            \coordinate (cc1) at ($(top)!.25!(bot)$);
            \coordinate (cc2) at ($(top)!.75!(bot)$);

            \node[above] at (c |- current bounding box.north) {\ref{grouplegend}};
            \node[below] at (cc1 |- current bounding box.south) {(a) High-contended YCSB (zipf=0.99)};
            \node[above] at (cc2 |- current bounding box.south) {(b) Medium-contended YCSB (zipf=0.6)};
        \end{tikzpicture}
    } \caption[YCSB small-transaction workloads results on fast storage]{YCSB
    small-transaction workloads results with 120 processing threads on fast
    storage. In write-intensive workloads, each transaction performs 8 reads and
    8 writes. In read-intensive workloads, each transaction performs 15 reads
    and 1 write. Note that, in the medium-contended workload, several variants
    have abort rates very close to 0.}
    \label{fig:ycsb:fast_storage}
\end{figure}

Figure~\ref{fig:ycsb:fast_storage} presents the goodput results for all our
timestamp-based concurrency control (CC) variants across four small-transaction
YCSB workloads when using fast storage. 

The first key result is that timestamp-based CC methods work much better with
fast storage than 2PL and KR-OCC. This improvement happens because
timestamp-based methods allow more transactions to run at the same time (higher
concurrency) and result in fewer aborts, whereas 2PL and KR-OCC start to
experience CPU limitations when storage is no longer slow. KR-OCC is especially
affected, since it needs more synchronization to check for conflicts when
committing transactions. 

The second main finding is that splitting timestamp storage using \sketchname is
effective. However, when Disk variants store timestamps on SplinterDB with fast
storage, they are still slower than pure in-memory dedicated timestamp storage.
This is because accessing timestamps through SplinterDB is more complex than
accessing them from memory directly.

A third observation is that \sketchname variants do not reach the performance of
the idealized Memory counterpart as storage becomes faster. This is due to the
overhead from the \sketchname data structure itself. On slow storage, these
overheads are negligible compared to disk I/O latency, but on fast storage they
become significant. The overhead has several components: (1) Hash table operations:
\sketchname must create and delete entries in a hash table, which requires 
allocating and copying memory for keys based on their reference count. In the 
high-contention, read-intensive workload for TicToc-\psketchname, roughly 80\% of 
\texttt{IncRef} calls are responsible for creating new entries in the hash table, 
which increases overhead. (2) Sketch eviction: \fsketchname has additional overhead 
because it needs to evict keys from the hash table to the sketch when reference 
counts reach zero, a step that does not occur in the \psketchname variant. (3) 
Memory allocation: Both variants require dynamic memory allocation for hash table 
entries and key storage, which incurs CPU cycles and potential cache misses. These 
overheads collectively create a performance gap between Memory and \sketchname 
variants that grows larger on fast storage, where CPU cycles become the limiting 
factor rather than storage latency.

In the high-contention, read-intensive workload, TicToc-\psketchname is
3.55$\times$ and 21.5$\times$ faster than 2PL and KR-OCC, respectively.
STO-\psketchname is 1.39$\times$ faster than 2PL and 8.1$\times$ faster than
KR-OCC. TicToc-\psketchname increases goodput by 286\% compared to TicToc-Disk.
STO-\psketchname increases goodput by 139\% compared to STO-Disk. However,
compared to the Memory variant, TicToc-\psketchname has 39.5\% lower goodput,
and STO-\psketchname is 22.5\% lower.

For the write-intensive workload, similar trends appear but the differences are
smaller. This is because the storage latency is less important for write
operations in write-optimized data structures like LSM-trees. In the
high-contention, write-intensive case, TicToc-\psketchname is 2.12$\times$ and
3.16$\times$ faster than 2PL and KR-OCC, respectively. STO-\psketchname is
closer to 2PL and 1.53$\times$ faster than KR-OCC. TicToc-\psketchname shows a
200\% improvement over TicToc-Disk, and STO-\psketchname improves by 72.2\% over
STO-Disk.

The \fsketchname variant removes the I/O overhead from accessing timestamps on
disk and so reaches higher goodput than 2PL and KR-OCC. However, it still has
less goodput than \psketchname variants because it needs to do extra work to
evict keys and also has a higher abort rate.

In workloads with medium contention, these overheads play a clearer role. Since
there is less contention, the protocol algorithm itself, rather than transaction
retries, affects goodput more strongly. In these cases, 2PL slightly outperforms
\sketchname variants because: (1) with reduced contention, lock conflicts become
rare, minimizing 2PL's traditional weakness; (2) 2PL avoids the hash table
operations and memory allocation overheads inherent to \sketchname; and (3) the
relative cost of lock acquisition and release becomes smaller compared to
\sketchname's timestamp management operations when storage is fast. This result
demonstrates that while \sketchname provides substantial benefits, there is room
for optimization to reduce overhead, particularly for medium-contention workloads
on fast storage.



\subsubsection{YCSB Mixed-Transaction Workloads}


\begin{figure}[!t]
    \centering
    \resizebox{\textwidth}{!}{%
        \begin{tikzpicture}
            \begin{groupplot}[group style={group size=4 by 1,
                            /pgf/bar width=2.7pt},
                    height = 3.6cm,
                    width = 4.5cm,
                    ybar= 2*\pgflinewidth,
                    xtick={-1, 0.425, 3.1},
                    xticklabels={2PL/OCC, STO, TicToc},
                    ymajorgrids=true,
                    enlarge x limits = 0.225,
                    legend columns=-1,
                    legend entries={{\ssmall Memory (Idealized)}, {\ssmall Disk}, {\ssmall Disk-Cache}, {\ssmall \fsketchname},  {\ssmall \psketchname}},
                    area legend,
                    legend to name=grouplegend,
                ]

                % graph [1,1] high
                \nextgroupplot[YCSBThroughputBarChartPlot, ylabel={Goodput (KTPS)}, ylabel style={yshift=-5pt}]
                \addplot[2PL-BarStyle, forget plot] coordinates {(-0.5, 174.950333)};
                \addplot[KR-OCC-BarStyle, forget plot] coordinates {(-0.26, 9.13111)};
                \addplot[Memory-BarStyle] coordinates {(0.425, 180.33) (3, 311.536)};
                \addplot[Disk-BarStyle] coordinates {(0.425, 92.825067) (3, 125.022333)};
                \addplot[Disk-Cache-BarStyle] coordinates {(0.425, 64.094933) (3, 91.446867)};
                \addplot[Counter-Lazy-BarStyle] coordinates {(0.425, 71.368433) (3, 105.553667)};
                \addplot[FPSketch-BarStyle] coordinates {(0.425, 140.384333) (3, 221.610333)};
                \coordinate (top) at (rel axis cs:0,1);% coordinate at top of the first plot

                % graph [2,1] abort rate
                \nextgroupplot[AbortRateBarChartPlot, ylabel={Abort rate}, ylabel style={yshift=-3pt}]
                \addplot[2PL-BarStyle, forget plot] coordinates {(-0.5, 0.138208)};
                \addplot[KR-OCC-BarStyle, forget plot] coordinates {(-0.26, 0.263853)};
                \addplot[Memory-BarStyle] coordinates {(0.425, 0.126144) (3, 0.058634)};
                \addplot[Disk-BarStyle] coordinates {(0.425, 0.122566) (3, 0.057911)};
                \addplot[Disk-Cache-BarStyle] coordinates {(0.425, 0.136498) (3, 0.059566)};
                \addplot[Counter-Lazy-BarStyle] coordinates {(0.425, 0.239943) (3, 0.117789)};
                \addplot[FPSketch-BarStyle] coordinates {(0.425, 0.139826) (3, 0.065083)};

                % graph [3,1] medium
                \nextgroupplot[YCSBThroughputBarChartPlot, ylabel={Goodput (KTPS)}, ylabel style={yshift=-8pt}]
                \addplot[2PL-BarStyle, forget plot] coordinates {(-0.5, 556.961)};
                \addplot[KR-OCC-BarStyle, forget plot] coordinates {(-0.26, 7.140063)};
                \addplot[Memory-BarStyle] coordinates {(0.425, 720.277667) (3, 714.753333)};
                \addplot[Disk-BarStyle] coordinates {(0.425, 345.637667) (3, 267.104333)};
                \addplot[Disk-Cache-BarStyle] coordinates {(0.425, 135.666333) (3, 134.658333)};
                \addplot[Counter-Lazy-BarStyle] coordinates {(0.425, 68.007267) (3, 284.942667)};
                \addplot[FPSketch-BarStyle] coordinates {(0.425, 315.562667) (3, 475.911667)};

                % graph [4,1] medium abort rate
                \nextgroupplot[AbortRateBarChartPlot, ylabel={Abort rate}, ylabel style={yshift=-3pt}]
                \addplot[2PL-BarStyle, forget plot] coordinates {(-0.5, 0.000335)};
                \addplot[KR-OCC-BarStyle, forget plot] coordinates {(-0.26, 0.000148)};
                \addplot[Memory-BarStyle] coordinates {(0.425, 0.000432) (3, 0.000202)};
                \addplot[Disk-BarStyle] coordinates {(0.425, 0.000239) (3, 0.000221)};
                \addplot[Disk-Cache-BarStyle] coordinates {(0.425, 0.000194) (3, 0.000078)};
                \addplot[Counter-Lazy-BarStyle] coordinates {(0.425, 0.257643) (3, 0.00023)};
                \addplot[FPSketch-BarStyle] coordinates {(0.425, 0.08147) (3, 0.000191)};
                \coordinate (bot) at (rel axis cs:1,0);% coordinate at bottom of the last plot

            \end{groupplot}
            \coordinate (c) at ($(top)!.5!(bot)$);
            \coordinate (cc1) at ($(top)!.25!(bot)$);
            \coordinate (cc2) at ($(top)!.75!(bot)$);

            \node[above] at (c |- current bounding box.north) {\ref{grouplegend}};
            \node[below] at (cc1 |- current bounding box.south) {(a) High-contended (zipf=0.99)};
            \node[above] at (cc2 |- current bounding box.south) {(b) Medium-contended (zipf=0.6)};
        \end{tikzpicture}
    } \caption[YCSB mixed-transaction workloads results on fast storage]{YCSB
    mixed-transaction workloads results with 120 processing threads on fast
    storage. 80\% of transactions perform 2 reads and 2 writes, and 20\% of
    transactions perform 28 reads. Note that, in the medium-contended workload,
    several variants have abort rates very close to 0.}
    \label{fig:ycsb:mixed_fast}
\end{figure}

\Cref{fig:ycsb:mixed_fast} presents the goodput outcomes for all timestamp-based
CC mechanisms across two mixed-transaction YCSB workloads under fast storage
conditions.

In the high-contention scenario, the trends of the different CC mechanisms are
similar to what we observed in the small-transaction workloads, but the
performance gaps are even larger. This is because mixed-transaction workloads
have a higher fraction of short transactions compared to the small-transaction
workloads. Looking at the numbers, TicToc-\psketchname achieves 1.78$\times$ and
24.3$\times$ higher throughput than 2PL and KR-OCC, respectively.
STO-\psketchname achieves 1.13$\times$ and 15.37$\times$ higher throughput than
2PL and KR-OCC, respectively. TicToc-\psketchname also reaches 77\% higher
goodput than TicToc-Disk, and STO-\psketchname is 51\% faster than STO-Disk.
However, when compared against the Memory configuration, both
TicToc-\psketchname and STO-\psketchname have 28\% and 22\% lower goodput,
respectively.

For the medium-contended scenario, there are two main points. First, similar to
the small-transaction experiments, 2PL still outperforms the \sketchname
variants, although the timestamp-based CC mechanisms running with Memory
configuration are still able to outperform 2PL. This shows that timestamping
itself is a strong approach. Second, in this situation, a Disk-based
configuration can outperform its \sketchname variant: STO-Disk achieves
1.09$\times$ faster than STO-\psketchname, while TicToc-\psketchname still
outperforms TicToc-Disk. This is the result of differences in abort handling:
STO aborts transactions immediately upon timestamp order break, while TicToc
waits until commit time to make this decision. This leads to more
over-approximation of timestamps in \sketchname for STO, especially since
mixed-transaction workloads have more short transactions. With bigger sketch
size than the evaluation setup, the abort rate of STO-\psketchname decreases.
For example, STO-\psketchname with 256KiB sketch size achieves approximately 520
KTPS, improving goodput by 50.4\% compared to STO-Disk. Thus, the more memory
for sketch is necessary for STO if transactions are short and their operations
are executed quickly.


\subsubsection{YCSB Long-Transaction Workload}


\begin{figure}[!t]
    \centering
    \resizebox{.75\textwidth}{!}{%
        \begin{tikzpicture}
            \begin{groupplot}[group style={
                            group size=2 by 1,
                            horizontal sep=1.5cm,
                            /pgf/bar width=2.7pt
                        },
                    height = 3.6cm,
                    width = 4.5cm,
                    ybar= 2*\pgflinewidth,
                    xtick={-1, 0.425, 3.1},
                    xticklabels={2PL/OCC, STO, TicToc},
                    ymajorgrids=true,
                    enlarge x limits = 0.225,
                    legend columns=1,
                    legend entries={{\ssmall Memory (Idealized)}, {\ssmall Disk}, {\ssmall Disk-Cache}, {\ssmall \fsketchname},  {\ssmall \psketchname}},
                    area legend,
                    legend to name=grouplegend,
                ]

                % graph [1,1] high
                \nextgroupplot[YCSBThroughputBarChartPlot, ylabel={Goodput (KTPS)}, ylabel style={yshift=-5pt}]
                \addplot[2PL-BarStyle, forget plot] coordinates {(-0.5, 7.24927)};
                \addplot[KR-OCC-BarStyle, forget plot] coordinates {(-0.26, 0.275767)};
                \addplot[Memory-BarStyle] coordinates {(0.425, 14.4813) (3, 9.571037)};
                \addplot[Disk-BarStyle] coordinates {(0.425, 13.3011) (3, 9.311203)};
                \addplot[Disk-Cache-BarStyle] coordinates {(0.425, 12.304867) (3, 8.088823)};
                \addplot[Counter-Lazy-BarStyle] coordinates {(0.425, 13.978767) (3, 9.499523)};
                \addplot[FPSketch-BarStyle] coordinates {(0.425, 14.869067) (3, 9.124267)};
                \coordinate (top) at (rel axis cs:0,1);% coordinate at top of the first plot

                % graph [1,2] abort rate
                \nextgroupplot[AbortRateBarChartPlot, ylabel={Abort rate}, ylabel style={yshift=-5pt}]
                \addplot[2PL-BarStyle, forget plot] coordinates {(-0.5, 0.542658)}; % double check
                \addplot[KR-OCC-BarStyle, forget plot] coordinates {(-0.26, 0.829988)}; % double check
                \addplot[Memory-BarStyle] coordinates {(0.425, 0.260036) (3, 0.24408)}; % double check
                \addplot[Disk-BarStyle] coordinates {(0.425, 0.29484) (3, 0.248819)}; % double check sto
                \addplot[Disk-Cache-BarStyle] coordinates {(0.425, 0.307315) (3, 0.256368)}; % double check sto
                \addplot[Counter-Lazy-BarStyle] coordinates {(0.425, 0.267667) (3, 0.246097)}; % double check
                \addplot[FPSketch-BarStyle] coordinates {(0.425, 0.263106) (3, 0.247183)}; % double check
                \coordinate (bot) at (rel axis cs:1,0);% coordinate at bottom of the last plot

            \end{groupplot}
            \coordinate (c) at ($(top)!.5!(bot)$);

            \node[right=4cm] at (c |- current bounding box.east) {\ref{grouplegend}};
        \end{tikzpicture}
    } \caption[YCSB long-transaction workloads results on fast storage]{YCSB
    long-transaction workloads results with 120 processing threads on fast
    storage. 5\% of transactions perform 1000 reads and the rest of transactions
    perform 8 writes and 8 reads. The distribution is Zipfian with 0.9.}
    \label{fig:ycsb:long_fast}
\end{figure}

The results for long-transaction workloads in \Cref{fig:ycsb:long_fast} are
similar to those on other types of storage. On fast storage, STO and TicToc with
approximate timestamp storage reach goodput almost equal to their Memory
versions. This shows that approximate timestamps work well for both large and
small transactions on fast storage as well.

\subsubsection{TPC-C Workloads}

\begin{figure*}[!t]
    \centering
    \resizebox{\textwidth}{!}{%
        \begin{tikzpicture}
            \begin{groupplot}[group style={group size=4 by 2,
                            /pgf/bar width=2.7pt},
                    height = 3.6cm,
                    width = 4.5cm,
                    ybar= 2*\pgflinewidth,
                    xtick={-0.8625, 0.425, 3},
                    xticklabels={2PL/OCC, STO, TicToc},
                    xticklabel style={font=\tiny},
                    ymajorgrids=true,
                    enlarge x limits = 0.2,
                    legend columns=-1,
                    legend entries={{\ssmall Memory (Idealized)}, {\ssmall Disk}, {\ssmall Disk-Cache}, {\ssmall \fsketchname},  {\ssmall \psketchname}},
                    area legend,
                    legend to name=grouplegend,
                ]

                % graph [1,1]
                \nextgroupplot[title={4 warehouses},YCSBThroughputBarChartPlot, ylabel={Goodput (KTPS)}]
                \addplot[2PL-BarStyle, forget plot] coordinates {(-0.5, 25.981933)};
                \addplot[KR-OCC-BarStyle, forget plot] coordinates {(-0.26, 4.766923)};
                \addplot[Memory-BarStyle] coordinates {(0.425, 59.3193) (3, 54.249267)};
                \addplot[Disk-BarStyle] coordinates {(0.425, 23.122733) (3, 29.668767)};
                \addplot[Disk-Cache-BarStyle] coordinates {(0.425, 17.227167) (3, 19.149033)};
                \addplot[Counter-Lazy-BarStyle] coordinates {(0.425, 41.3587) (3, 27.051667)};
                \addplot[FPSketch-BarStyle] coordinates {(0.425, 56.8103) (3, 42.8906)};
                \coordinate (top) at (rel axis cs:0,1);% coordinate at top of the first plot

                % graph [1,2]
                \nextgroupplot[title={8 warehouses}, YCSBThroughputBarChartPlot]
                \addplot[2PL-BarStyle, forget plot] coordinates {(-0.5, 40.674833)};
                \addplot[KR-OCC-BarStyle, forget plot] coordinates {(-0.26, 2.690287)};
                \addplot[Memory-BarStyle] coordinates {(0.425, 95.317467) (3, 85.340067)};
                \addplot[Disk-BarStyle] coordinates {(0.425, 33.973233) (3, 41.828467)};
                \addplot[Disk-Cache-BarStyle] coordinates {(0.425, 29.014767) (3, 25.245733)};
                \addplot[Counter-Lazy-BarStyle] coordinates {(0.425, 42.335667) (3, 30.594867)};
                \addplot[FPSketch-BarStyle] coordinates {(0.425, 85.059467) (3, 58.450233)};

                % graph [1,3]
                \nextgroupplot[title={16 warehouses},YCSBThroughputBarChartPlot]
                \addplot[2PL-BarStyle, forget plot] coordinates {(-0.5, 60.0792)};
                \addplot[KR-OCC-BarStyle, forget plot] coordinates {(-0.26, 1.714923)};
                \addplot[Memory-BarStyle] coordinates {(0.425, 126.781667) (3, 113.534667)};
                \addplot[Disk-BarStyle] coordinates {(0.425, 49.8068) (3, 62.967367)};
                \addplot[Disk-Cache-BarStyle] coordinates {(0.425, 33.078067) (3, 31.654267)};
                \addplot[Counter-Lazy-BarStyle] coordinates {(0.425, 37.753067) (3, 38.951133)};
                \addplot[FPSketch-BarStyle] coordinates {(0.425, 111.767) (3, 78.6518)};


                % graph [1,4] 32wh
                \nextgroupplot[title={32 warehouses},YCSBThroughputBarChartPlot]
                \addplot[2PL-BarStyle, forget plot] coordinates {(-0.5, 91.371733)};
                \addplot[KR-OCC-BarStyle, forget plot] coordinates {(-0.26, 1.768823)};
                \addplot[Memory-BarStyle] coordinates {(0.425, 136.254) (3, 128.511333)};
                \addplot[Disk-BarStyle] coordinates {(0.425, 60.4086) (3, 77.2502)};
                \addplot[Disk-Cache-BarStyle] coordinates {(0.425, 35.204067) (3, 35.5132)};
                \addplot[Counter-Lazy-BarStyle] coordinates {(0.425, 54.0724) (3, 62.5415)};
                \addplot[FPSketch-BarStyle] coordinates {(0.425, 124.853) (3, 103.150333)};

                \coordinate (bot) at (rel axis cs:1,0);% coordinate at bottom of the last plot

                % graph [2,1] 4 wh
                \nextgroupplot[AbortRateBarChartPlot, ylabel={Abort rate}]
                \addplot[2PL-BarStyle, forget plot] coordinates {(-0.5, 0.343923)};
                \addplot[KR-OCC-BarStyle, forget plot] coordinates {(-0.26, 0.582189)};
                \addplot[Memory-BarStyle] coordinates {(0.425, 0.308396) (3, 0.358287)};
                \addplot[Disk-BarStyle] coordinates {(0.425, 0.300735) (3, 0.362429)};
                \addplot[Disk-Cache-BarStyle] coordinates {(0.425, 0.306576) (3, 0.373073)};
                \addplot[Counter-Lazy-BarStyle] coordinates {(0.425, 0.378139) (3, 0.439741)};
                \addplot[FPSketch-BarStyle] coordinates {(0.425, 0.316455) (3, 0.38327)};

                % graph [2,2] 8 wh
                \nextgroupplot[AbortRateBarChartPlot]
                \addplot[2PL-BarStyle, forget plot] coordinates {(-0.5, 0.26031)};
                \addplot[KR-OCC-BarStyle, forget plot] coordinates {(-0.26, 0.619373)};
                \addplot[Memory-BarStyle] coordinates {(0.425, 0.259365) (3, 0.328457)};
                \addplot[Disk-BarStyle] coordinates {(0.425, 0.249634) (3, 0.322022)};
                \addplot[Disk-Cache-BarStyle] coordinates {(0.425, 0.259498) (3, 0.320605)};
                \addplot[Counter-Lazy-BarStyle] coordinates {(0.425, 0.383305) (3, 0.448231)};
                \addplot[FPSketch-BarStyle] coordinates {(0.425, 0.279779) (3, 0.364156)};

                % graph [2,3] 16 wh
                \nextgroupplot[AbortRateBarChartPlot]
                \addplot[2PL-BarStyle, forget plot] coordinates {(-0.5, 0.181629)};
                \addplot[KR-OCC-BarStyle, forget plot] coordinates {(-0.26, 0.626907)};
                \addplot[Memory-BarStyle] coordinates {(0.425, 0.194619) (3, 0.282973)};
                \addplot[Disk-BarStyle] coordinates {(0.425, 0.189055) (3, 0.271532)};
                \addplot[Disk-Cache-BarStyle] coordinates {(0.425, 0.180874) (3, 0.280659)};
                \addplot[Counter-Lazy-BarStyle] coordinates {(0.425, 0.572401) (3, 0.40728)};
                \addplot[FPSketch-BarStyle] coordinates {(0.425, 0.23819) (3, 0.320579)};

                % graph [2,4] 32 wh
                \nextgroupplot[AbortRateBarChartPlot]
                \addplot[2PL-BarStyle, forget plot] coordinates {(-0.5, 0.133359)};
                \addplot[KR-OCC-BarStyle, forget plot] coordinates {(-0.26, 0.512981)};
                \addplot[Memory-BarStyle] coordinates {(0.425, 0.040332) (3, 0.202382)};
                \addplot[Disk-BarStyle] coordinates {(0.425, 0.067348) (3, 0.215892)};
                \addplot[Disk-Cache-BarStyle] coordinates {(0.425, 0.063018) (3, 0.20933)};
                \addplot[Counter-Lazy-BarStyle] coordinates {(0.425, 0.74949) (3, 0.291205)};
                \addplot[FPSketch-BarStyle] coordinates {(0.425, 0.167312) (3, 0.247521)};
            \end{groupplot}
            \coordinate (c) at ($(top)!.5!(bot)$);

            \node[above] at (c |- current bounding box.north) {\ref{grouplegend}};
        \end{tikzpicture}
    } \caption[TPC-C results on fast storage]{TPC-C results with 120 processing
    threads on fast storage (More warehouses means less contention).}
    \label{fig:tpcc_fast}
\end{figure*}

\Cref{fig:tpcc_fast} shows the TPC-C benchmark results on fast storage. The
figure includes both goodput and abort rates for different concurrency control
methods as we change the number of warehouses (which also changes the
contention level). Similar to the YCSB results, both \psketchname and
\fsketchname always have lower goodput than the Memory version because of extra
overheads, but they are still much better than the Disk-based methods.

TicToc-\psketchname achieves up to 9 to 58.3$\times$ higher goodput than
2PL/KR-OCC, and about 25--44.5\% higher goodput than TicToc-Disk at all
contention levels. STO-\psketchname achieves up to 11.9 to 70.6$\times$ more
goodput than 2PL/KR-OCC, and about 105--150\% improvement over STO-Disk in all
cases. The TPC-C benchmark consists of both short and long transactions, and STO
achieves higher goodput than TicToc for long transactions. This matches our
earlier findings from the YCSB long-transaction workload in
\Cref{fig:ycsb:long_fast}.

Comparing \psketchname and \fsketchname, we see that \psketchname always has
better goodput at every contention level. This is because \psketchname has less
overhead from sketch evictions. However, the gap between the two gets smaller
when the workload is less contentious (that is, when there are more warehouses).
This indicates that \psketchname is a good option for deployment on fast
storage.


\section{Summary}

This chapter presents a comprehensive evaluation of \sketchname across diverse
storage media, from traditional slow storage (SATA SSD and HDD) to modern fast
storage emulating CXL-based flash with DRAM-like latency. Our experimental study
reveals several important findings and lessons that guide the adoption and future
development of \sketchname.

\subsection{Key Findings}

Our evaluation demonstrates that \sketchname remains effective across a wide
range of storage technologies. On slow storage media, including SATA SSD and
HDD, \sketchname variants consistently achieve goodput close to the idealized
Memory configuration. The benefits are most pronounced in high-contention,
write-intensive workloads, where \sketchname improves goodput by up to 569\%
compared to Disk-based timestamp storage on SATA SSD and up to 519\% on HDD.
Even in medium-contention scenarios where storage I/O becomes the primary
bottleneck, \sketchname still provides substantial improvements by eliminating
the overhead of accessing timestamps from disk.

On fast storage media, our experiments reveal a different set of
characteristics. Timestamp-based concurrency control methods, when combined with
\sketchname, significantly outperform 2PL and KR-OCC because they enable higher
concurrency and result in fewer transaction aborts. For instance,
TicToc-\psketchname achieves up to 3.55$\times$ and 21.5$\times$ higher goodput
than 2PL and KR-OCC, respectively, in high-contention, read-intensive workloads.
However, \sketchname variants do not reach the performance of the idealized
Memory configuration due to overhead from their operations, memory allocation,
and key management in the \sketchname data structure itself.

\subsection{Lessons Learned}

Our experimental study yields several important lessons. First, the effectiveness
of \sketchname varies with storage characteristics. On slow storage, where disk
I/O dominates performance, \sketchname provides substantial benefits by removing
the need to access timestamps from persistent storage. On fast storage, while
\sketchname still outperforms Disk-based approaches, the relative overhead of
the \sketchname data structure becomes more visible. This suggests that future
optimizations should focus on reducing this overhead, particularly for
high-performance storage environments.

Second, we observe that \psketchname consistently outperforms \fsketchname
across all storage types and workloads. The additional overhead of evicting keys
to the sketch in \fsketchname creates a performance gap that becomes more
pronounced as storage becomes faster. Moreover, \psketchname also outperforms
Disk-based configurations in the vast majority of scenarios, making it the clear
preferred choice among all timestamp storage options. Only in rare cases with
medium-contention scenarios on fast storage can Disk-based configurations
occasionally match or slightly exceed \psketchname performance, when the
overhead of maintaining the sketch outweighs the benefits of avoiding disk
accesses. In practice, these cases are limited, and \psketchname offers superior
performance while still eliminating the need to access timestamps from
persistent storage.

Third, our results highlight the importance of transaction characteristics in
determining \sketchname's effectiveness. Short transactions with quick
operations benefit more from \sketchname, especially on slow storage. For long
transactions, the impact is smaller because the transaction execution time
dominates, but \sketchname still provides improvements by reducing timestamp
access overhead. The mixed-transaction workloads demonstrate that \sketchname
maintains its effectiveness even when workloads contain both short and long
transactions.

Finally, our evaluation reveals that storage media characteristics fundamentally
change the performance trade-offs in concurrency control. As storage becomes
faster, timestamp-based methods with \sketchname emerge as the clear winners
over traditional locking-based approaches. This finding suggests that as storage
technology continues to evolve toward lower latency and higher bandwidth,
\sketchname will become increasingly important for enabling high-performance
transaction processing in disk-based key-value stores.

\subsection{Deployment Guidelines}

Based on our comprehensive evaluation, we offer the following deployment
guidelines for \sketchname:

\textbf{Variant selection:} \psketchname should be preferred over \fsketchname in
virtually all scenarios, as it consistently outperforms \fsketchname across all
storage types and workloads while requiring similar memory footprint. The only
exception is when memory constraints are extremely tight and the workload has very
low contention, where \fsketchname's ability to evict keys might provide marginal
memory savings.

\textbf{Storage-specific recommendations:} On slow storage (SATA SSD and HDD),
\sketchname provides dramatic improvements and should be adopted whenever 
timestamp-based CC methods are used. The benefits are most pronounced for 
high-contention workloads. On fast storage (CXL-based flash or similar), 
\sketchname remains beneficial, particularly for high-contention scenarios where
timestamp-based methods significantly outperform 2PL and KR-OCC. However, the
overhead becomes more visible, suggesting future optimization opportunities.

\textbf{Workload considerations:} \sketchname is most beneficial for workloads with
short transactions and high contention, especially write-intensive operations. For
long transactions, \sketchname still provides improvements, but the impact is
smaller since transaction execution time dominates. Mixed workloads benefit
substantially from \sketchname across all storage types.

\textbf{Memory requirements:} \sketchname requires minimal memory (32KiB in our
experiments with an 80GB database), making it practical for deployment even in
memory-constrained environments. This footprint remains consistent across storage
types, as the sketch size is workload-dependent rather than storage-dependent.

\textbf{Concurrency control pairing:} TicToc with \psketchname consistently
provides the best performance across all evaluated scenarios, making it the
recommended combination for new deployments. STO with \psketchname is also
excellent, particularly for long transactions. MVTO can benefit from \sketchname
on slow storage, but may face memory limitations on fast storage with 
write-intensive workloads.

In conclusion, \sketchname proves to be a valuable technique for enabling modern
timestamp-based concurrency control methods on persistent storage across a wide
spectrum of storage technologies. While there are opportunities for further
optimization, particularly for fast storage environments, the experimental
evidence strongly supports the adoption of \sketchname in production systems
that use various storage media.
