\newtheorem{theorem}{Theorem}
\newtheorem{lemma}[theorem]{Lemma}
\newtheorem{property}[theorem]{Property}

\newcommand{\defn}[1]{\textbf{\textit{#1}}\xspace}

\newcommand{\bets}{B\texorpdfstring{$^\varepsilon$}{\^{}e}-trees\xspace}
\newcommand{\btree}{B-tree\xspace}
\newcommand{\btrees}{B-trees\xspace}

\newcommand{\Sketchname}{FPSketch\xspace}
\newcommand{\sketchname}{\Sketchname}
\newcommand{\Fsketchname}{Focus-Counter\xspace}
\newcommand{\fsketchname}{\Fsketchname}
\newcommand{\Psketchname}{Focus-Sketch\xspace}
\newcommand{\psketchname}{\Psketchname}

\algdef{SE}% flags used internally to indicate we're defining a new block statement
[STRUCT]% new block type, not to be confused with loops or if-statements
{Struct}% "\Struct{name}" will indicate the start of the struct declaration
{EndStruct}% "\EndStruct" ends the block indent
[1]% There is one argument, which is the name of the data structure
{\textbf{struct} \textsc{#1}}% typesetting of the start of a struct
{\textbf{end struct}}% typesetting the end of the struct

\algblockdefx{Atomic}{EndAtomic}{\textbf{begin atomic section}}{\textbf{end atomic section}}

\newcommand{\declareKeyword}[1]{\expandafter\newcommand\csname #1\endcsname{\textsc{#1}\xspace}}
\declareKeyword{lock}
\declareKeyword{TryLock}
\declareKeyword{unlock}
\declareKeyword{abort}
\declareKeyword{locked}
\declareKeyword{unlocked}

\newcommand{\rts}[1]{#1.\textit{rts}\xspace}
\newcommand{\wts}[1]{#1.\textit{wts}\xspace}
\newcommand{\val}[1]{#1.\textit{value}\xspace}
\newcommand{\tuple}[1]{#1.\textit{tuple}\xspace}
\newcommand{\key}[1]{#1.\textit{key}\xspace}
\newcommand{\tuplerts}[1]{\tuple{#1}.\textit{rts}\xspace}
\newcommand{\tuplewts}[1]{\tuple{#1}.\textit{wts}\xspace}
\newcommand{\tupleval}[1]{\tuple{#1}.\textit{value}\xspace}
\newcommand{\dbread}[1]{\fcolorbox{green}{white}{#1}}
\newcommand{\dbwrite}[1]{\fcolorbox{red}{white}{#1}}
\newcommand{\timestamps}[1]{#1.\textit{timestamps}\xspace}

\newcommand{\commitTime}{\ensuremath{\tau}\xspace}
\newcommand{\Assign}[2]{\State #1 $\leftarrow$ #2}

\newcommand{\DeclareVar}[3]{\State \textbf{#1} \textit{#2}#3}

\newcommand{\hashtable}[1]{\textit{hashtable}\xspace}
\newcommand{\hashtableput}[2]{\textit{hashtable}.\textsc{Put}(#1, #2)\xspace}
\newcommand{\hashtableget}[1]{\textit{hashtable}.\textsc{GetItemPtr}(#1)\xspace}
\newcommand{\hashtableremove}[1]{\textit{hashtable}.\textsc{Remove}(#1)\xspace}
\newcommand{\sketchput}[2]{\textit{sketch}.\textsc{Put}(#1, #2)\xspace}
\newcommand{\sketchget}[1]{\textit{sketch}.\textsc{Get}(#1)\xspace}
\newcommand{\timestampmax}{\textsc{TimestampMax}\xspace}
\newcommand{\timestampmin}{\textsc{TimestampMin}\xspace}
\newcommand{\timestampmaxfunc}[2]{\timestampmax(#1, #2)\xspace}
\newcommand{\timestampminfunc}[2]{\timestampmin(#1, #2)\xspace}
\newcommand{\refcount}[1]{#1.\textit{refcount}\xspace}
\newcommand{\hash}[2]{\textsc{Hash}(#1, #2)\xspace}
\newcommand{\atomicload}[1]{\textsc{AtomicLoad}(#1)\xspace}
\newcommand{\atomicfetchadd}[2]{\textsc{AtomicFetchAdd}(#1, #2)\xspace}
\newcommand{\atomiccas}[3]{\textsc{AtomicCompareExchange}(#1, #2, #3)\xspace}
\newcommand{\tsdelta}[1]{#1.\textit{delta}\xspace}
\newcommand{\dirtybit}[1]{#1.\textit{dirty\_bit}\xspace}



\newcommand{\linestylekey}[1]{#1}

\pgfplotsset{
    every mark/.append style={solid},
}

\pgfplotsset{
    LineStyle/.style={
        mark size=1.5pt,
        solid,
    },
    Disk-LineStyle/.style={
        LineStyle,
        color=red,
        mark=pentagon,
    },
    Disk-Cache-LineStyle/.style={
        LineStyle,
        color=blue,
        mark=+,
    },
    Counter-Lazy-LineStyle/.style={
        LineStyle,
        color=teal,
        mark=triangle,
    },
    Memory-LineStyle/.style={
        LineStyle,
        color=black,
        mark=x,
    },
    FPSketch-LineStyle/.style={
        LineStyle,
        color=violet,
        mark=o,
    },
}

\pgfplotsset{
    LineStyle/.style={
      mark size=1.5pt,
    },
    2PL-LineStyle/.style={
      LineStyle,
      color=teal,
      mark=triangle,
      solid,
    },
    KR-OCC-LineStyle/.style={
      LineStyle,
      color=teal,
      mark=diamond,
      solid,
    },
    KR-OCC-Serial-LineStyle/.style={
      LineStyle,
      color=teal,
      mark=diamond,
      solid,
    },
    KR-OCC-Parallel-LineStyle/.style={
      LineStyle,
      color=teal,
      mark=square,
      solid,
    },
    STO-Disk-LineStyle/.style={
      LineStyle,
      color=red,
      mark=pentagon,
      solid,
    },
    STO-Disk-Cache-LineStyle/.style={
      LineStyle,
      color=red,
      mark=+,
      solid,
    },
    STO-Memory-LineStyle/.style={
      LineStyle,
      color=red,
      mark=x,
      solid,
    },
    STO-FPSketch-LineStyle/.style={
      LineStyle,
      color=red,
      mark=o,
      solid,
    },
    TicToc-Disk-LineStyle/.style={
      LineStyle,
      color=violet,
      mark=pentagon,
      solid,
    },
    TicToc-Disk-Cache-LineStyle/.style={
      LineStyle,
      color=violet,
      mark=+,
      solid,
    },
    TicToc-Memory-LineStyle/.style={
      LineStyle,
      color=violet,
      mark=x,
      solid,
    },
    TicToc-FPSketch-LineStyle/.style={
      LineStyle,
      color=violet,
      mark=o,
      solid,
    },
    MVTO-Disk-LineStyle/.style={
      LineStyle,
      color=olive,
      mark=pentagon,
      solid,
    },
    MVTO-Disk-Cache-LineStyle/.style={
      LineStyle,
      color=olive,
      mark=+,
      solid,
    },
    MVTO-Memory-LineStyle/.style={
      LineStyle,
      color=olive,
      mark=x,
      solid,
    },
    MVTO-FPSketch-LineStyle/.style={
      LineStyle,
      color=olive,
      mark=o,
      solid,
    },
}

\pgfplotsset{
    BarStyle/.style={
        },
    Disk-BarStyle/.style={
            BarStyle,
            pattern=horizontal lines,
            pattern color=red
        },
    Disk-Cache-BarStyle/.style={
            BarStyle,
            pattern=bricks,
            pattern color=blue
        },
    Counter-Lazy-BarStyle/.style={
            BarStyle,
            pattern=crosshatch,
            pattern color=teal
        },
    Memory-BarStyle/.style={
            BarStyle,
            fill=white
        },
    FPSketch-BarStyle/.style={
            BarStyle,
            fill=violet
        }
}

\pgfplotsset{
    BarStyle/.style={
        },
    2PL-BarStyle/.style={
            BarStyle,
            pattern=north west lines,
            pattern color=black,
        },
    KR-OCC-BarStyle/.style={
            BarStyle,
            pattern=north east lines,
            pattern color=black,
        },
    KR-OCC-Serial-BarStyle/.style={
            BarStyle,
            pattern=north east lines,
            pattern color=black
        },
    KR-OCC-Parallel-BarStyle/.style={
            BarStyle,
            pattern=north west lines,
            pattern color=black
        },
    STO-Disk-BarStyle/.style={
            BarStyle,
            pattern=grid,
            pattern color=teal
        },
    STO-Disk-Cache-BarStyle/.style={
            BarStyle,
            pattern=bricks,
            pattern color=teal
        },
    STO-Counter-Lazy-BarStyle/.style={
            BarStyle,
            pattern=north west lines,
            pattern color=teal
        },
    STO-Memory-BarStyle/.style={
            BarStyle,
            pattern=crosshatch,
            pattern color=teal
        },
    STO-FPSketch-BarStyle/.style={
            BarStyle,
            fill=teal
        },
    TicToc-Disk-BarStyle/.style={
            BarStyle,
            pattern=grid,
            pattern color=violet
        },
    TicToc-Disk-Cache-BarStyle/.style={
            BarStyle,
            pattern=bricks,
            pattern color=violet
        },
    TicToc-Counter-Lazy-BarStyle/.style={
            BarStyle,
            pattern=north west lines,
            pattern color=violet
        },
    TicToc-Memory-BarStyle/.style={
            BarStyle,
            pattern=crosshatch,
            pattern color=violet
        },
    TicToc-FPSketch-BarStyle/.style={
            BarStyle,
            fill=violet
        },
    MVTO-Disk-BarStyle/.style={
            BarStyle,
            pattern=grid,
            pattern color=olive
        },
    MVTO-Disk-Cache-BarStyle/.style={
            BarStyle,
            pattern=bricks,
            pattern color=olive
        },
    MVTO-Counter-Lazy-BarStyle/.style={
            BarStyle,
            pattern=north west lines,
            pattern color=olive
        },
    MVTO-Memory-BarStyle/.style={
            BarStyle,
            pattern=crosshatch,
            pattern color=olive
        },
    MVTO-FPSketch-BarStyle/.style={
            BarStyle,
            fill=olive
        },
}

\pgfplotsset{
    YCSBThroughputBarChartPlot/.style={
        label style={font=\footnotesize},
        ticklabel style={font=\ssmall},
        major x tick style = transparent,
        ymin=0,
    },
    AbortRateBarChartPlot/.style={
        ymin=0,
        label style={font=\footnotesize},
        ticklabel style={font=\ssmall},
        major x tick style = transparent,
    }
}