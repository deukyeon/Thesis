This dissertation addresses a fundamental challenge in modern database systems:
enabling efficient timestamp-based concurrency control in disk-based
transactional databases. Through both theoretical analysis and experimental
evaluation, the research demonstrates that approximate timestamp storage enables
high-performance transaction processing in disk-based databases across diverse
storage media technologies.

It presents two primary contributions. First, it introduces \sketchname, a novel
approximate timestamp storage system that enables efficient timestamp-based
concurrency control in disk-based databases. \sketchname combines a hash table
for exact timestamps of active keys with a sketch for approximate upper bounds
of inactive keys, achieving the performance benefits of fully in-memory
timestamp storage while requiring only minimal memory—as little as 32KiB for an
80GB database. Second, it presents a comprehensive analytical and experimental
evaluation demonstrating that \sketchname remains effective across a wide
spectrum of storage technologies, from traditional hard disk drives with
millisecond latencies to emerging CXL-based storage approaching DRAM-like
speeds.

\subsubsection{The \sketchname System}

The first contribution establishes \sketchname as a practical solution to the
metadata storage problem in timestamp-based concurrency control. Its design is
grounded in a key insight: for timestamp-based protocols like STO, MVTO, and
TicToc, overapproximating timestamps does not violate correctness—it may cause
harmless extra aborts but preserves serializability. This insight allows
utilizing approximate data structures while maintaining all correctness
guarantees.

\sketchname's hybrid architecture, combining a foveated region (hash table) and
peripheral region (sketch), ensures that active keys maintain exact timestamps
throughout their transaction lifetime while inactive keys can be safely
approximated. It formally proves that this approach satisfies the necessary
properties for correctness with all three protocols studied. The
evaluation demonstrates that \sketchname-based implementations achieve dramatic
performance improvements: TicToc with \psketchname improves goodput by up to
5.9$\times$ over disk-based timestamp storage, up to 14$\times$ over traditional
2PL, and reaches performance close to an idealized in-memory system.

\subsubsection{Broad Applicability Across the Storage Spectrum}

The second contribution establishes the universal applicability of the
approximate timestamp storage approach. This comprehensive evaluation across
HDDs, SATA SSDs, NVMe SSDs, and simulated CXL-based storage reveals that
\sketchname's benefits scale with the fundamental gap between local memory and
remote storage access. This finding ensures that \sketchname will remain
valuable as storage technology continues to evolve.

On slow storage (HDDs and SATA SSDs), where disk I/O dominates performance,
\sketchname beats traditional concurrency control methods like 2PL and KR-OCC by
eliminating the overhead of accessing timestamps from disk. TicToc-\psketchname
is up to 6.89$\times$ and 2.52$\times$ faster than 2PL and KR-OCC. \sketchname
shows the effectiveness in disk I/O reduction by improving goodput by as much as
569\% (SATA SSD) and 519\% (HDD) in write-intensive cases. Therefore,
\sketchname enables high-performance timestamp-based concurrency control on slow
storage.


On fast storage (simulated DRAM-like speed of CXL-based SSDs), the performance
characteristics undergo a fundamental shift: the system transitions from being
I/O-bound to CPU-bound as storage latencies approach the single-digit
microsecond range. This transition makes the overhead of \sketchname's data
structure operations (hash table management, memory allocation, and sketch
operations) more visible relative to storage access costs. However, \sketchname
continues to provide substantial benefits by eliminating timestamp access
overhead through persistent storage, and more importantly, timestamp-based
concurrency control methods combined with \sketchname significantly outperform
traditional approaches. Specifically, TicToc-\psketchname achieves up to
3.55$\times$ and 21.5$\times$ higher goodput than 2PL and KR-OCC, respectively,
in high-contention workloads on fast storage, demonstrating that timestamp-based
methods enable higher concurrency and fewer aborts when storage is no longer the
limiting factor. While \sketchname variants do not reach the performance of
idealized in-memory timestamp storage due to their internal overhead, they still
deliver dramatic improvements over disk-based approaches and traditional
concurrency control methods across the entire storage spectrum.

 
 

\section{Key Findings and Insights}

This research yields several important findings that guide both current
deployment and future development:

\subsubsection{Performance Characteristics}

Across all evaluated storage technologies and workloads, \sketchname
consistently outperforms traditional concurrency control methods and disk-based
timestamp storage. The \psketchname variant achieves goodput close to the
idealized Memory configuration while using only a tiny fraction of memory.
TicToc with \psketchname emerges as the clear performance leader, significantly
outperforming other concurrency control methods across all scenarios.

The memory efficiency of \sketchname is remarkable: a 32KiB sketch suffices for
an 80GB database, representing less than 0.00004\% of the database size. This
efficiency makes \sketchname practical even in memory-constrained environments
where storing all timestamps in RAM would be infeasible.

\subsubsection{Storage-Dependent Behavior}

The evaluation reveals that the nature of \sketchname's advantages changes
fundamentally as storage performance improves. On slow storage, traditional
concurrency control methods like 2PL and KR-OCC are commonly employed because
they do not depend on on-disk timestamp metadata; this avoids extra random I/O,
which is a major bottleneck on slow storage. \sketchname outperforms the
traditional CC methods by effectively eliminating I/O bottlenecks. On fast
storage, while \sketchname still effectively eliminates the overhead of
accessing timestamps from disk, the CPU overhead from \sketchname operations
becomes more visible, creating optimization opportunities for future work.

\subsubsection{Workload Sensitivity}

\sketchname demonstrates consistent effectiveness across diverse workloads, from
small transactions typical of OLTP systems to long transactions.
High-contention, write-intensive workloads show the largest improvements, but
even medium-contention scenarios benefit substantially. Mixed workloads
containing both short and long transactions maintain \sketchname's
effectiveness, demonstrating its robustness.

\subsubsection{Protocol Compatibility}

The analytical and experimental evaluations confirm that \sketchname correctly
integrates with STO, MVTO, and TicToc without any algorithmic changes to these
protocols. This plug-and-play compatibility makes adoption straightforward and
suggests that \sketchname could similarly integrate with current and future
timestamp-based concurrency control protocols.

\section{Broader Implications}

The success of \sketchname points to a broader principle: approximate metadata
management can enable high-performance system designs that would otherwise be
impractical. The key insight—that overapproximation preserves correctness for
many concurrency control protocols—may find application beyond timestamp
storage.

As storage technologies continue evolving toward faster, more memory-like
interfaces, the distinction between memory and storage blurs. \sketchname
demonstrates how application-specific caching strategies can bridge this gap,
providing a template for managing other types of frequently accessed metadata in
future systems.

The evaluation also highlights an important shift in database system design.
Traditional systems optimized for disk I/O as the dominant bottleneck. Modern
systems must optimize for CPU efficiency and memory hierarchy utilization.
\sketchname exemplifies this new design paradigm by prioritizing CPU and memory
efficiency while maintaining correctness guarantees.

\section{Future Work}

While \sketchname demonstrates strong performance across diverse storage
technologies, several directions offer opportunities for further improvement and
broader application.

\subsubsection{Optimizations for Fast Storage}

The evaluation reveals that on fast storage (CXL-based or similar), \sketchname
incurs CPU overhead from sketch operations, memory allocation, and key
management that prevents it from fully matching the idealized Memory
configuration. Future work could explore several optimization strategies:

First, \emph{lock-free and wait-free algorithms} could reduce synchronization
overhead in the sketch and hash table operations. Current implementations use
per-bucket locks which, while correct, may become bottlenecks on fast storage
where operations complete in nanoseconds.

Second, \emph{custom memory allocators} optimized for the access patterns of
\sketchname could reduce allocation overhead. Fast storage environments are
CPU-bound, making efficient memory management critical.

Third, \emph{hardware acceleration} could leverage modern CPU features like SIMD
instructions for bulk sketch operations or hardware transactional memory for
lock-free updates. Exploring how emerging CPU architectures can accelerate
\sketchname operations presents an interesting research direction.

Finally, \emph{adaptive sketch sizing} could dynamically adjust sketch size
based on workload characteristics, reducing overhead for low-contention
scenarios while maintaining accuracy for high-contention workloads.


\subsubsection{Evaluation on Real CXL-based SSDs}

While the evaluation includes a simulation of CXL-class latencies, an important
next step is to evaluate \sketchname on production CXL-connected devices. Real
hardware introduces effects that are difficult to capture in simulation,
including PCIe/CXL fabric contention, device firmware policies (e.g., interrupt
moderation, internal queueing, and thermal throttling), host driver and I/O
stack interactions, and tail-latency behaviors under bursty workloads. A
systematic experimental campaign on commercially available CXL-based SSDs and
memory expanders would (1) validate the analytical model and calibrate
constants, (2) quantify head- and tail-latency distributions and CPU overheads
at varying queue depths, NUMA placements, and PCIe topologies, and (3) surface
optimization opportunities specific to CXL (e.g., larger submission queues,
batching, and polling). These results would strengthen external validity and
refine guidance for deploying \sketchname on next-generation storage.


\subsubsection{Integration with Existing Database Systems}

While \sketchname is realized in SplinterDB, integration with other database
systems would validate broader applicability. Integrating \sketchname with other
database systems like LSM-tree based systems like RocksDB, B-tree systems like
PostgreSQL, or other database architectures could reveal system-specific
optimization opportunities.

\subsubsection{Beyond Timestamps: General Approximate Metadata}

The principle underlying \sketchname—that approximate metadata can preserve
correctness for certain protocols—may apply beyond timestamps. Could similar
techniques optimize storage of locks, version numbers, conflict detection
metadata, or other concurrency control state?

Exploring the space of metadata that can be safely approximated could yield
additional optimization opportunities. This direction requires identifying
metadata properties that allow approximation without violating correctness.

\subsubsection{Workload-Aware Optimization}

The evaluation demonstrates that \sketchname's effectiveness varies with
workload characteristics. Future work could develop \emph{adaptive \sketchname}
variants that automatically adjust their behavior based on observed workload
patterns. For example, sketch size could adapt to contention levels, eviction
policies could optimize for observed access patterns, or the hash table could
resize based on active key counts.

Machine learning techniques could potentially optimize \sketchname parameters
based on historical workload data, automatically tuning for best performance
without manual intervention.

\section{Concluding Remarks}

This dissertation demonstrates that approximate timestamp storage enables
efficient timestamp-based concurrency control in disk-based databases while
requiring minimal memory. Through the design and evaluation of \sketchname, the
research establishes that this approach remains effective across a wide spectrum
of storage technologies, from traditional hard drives to emerging memory-like
storage.

The key contribution is not merely a new data structure, but a
demonstration that approximate metadata management can unlock high-performance
system designs that would otherwise be impractical. As storage technology
continues evolving and the gap between memory and storage narrows, techniques
like \sketchname that optimize metadata access will become increasingly
important for achieving optimal database performance.

\sketchname represents a practical, deployable solution to a fundamental
challenge in modern database systems. By requiring only 32KiB of memory for an
80GB database while achieving performance close to idealized in-memory systems,
\sketchname makes advanced concurrency control protocols accessible to
real-world database deployments. The universal applicability of \sketchname
across storage technologies ensures its relevance both today and as storage
continues evolving.

More broadly, the dissertation contributes to a paradigm shift in transactional
system design, from optimizing for disk I/O to optimizing for CPU efficiency and
memory hierarchy utilization. As this shift continues, approximate metadata
management techniques will play an increasingly central role in high-performance
transaction processing in disk-based databases.

