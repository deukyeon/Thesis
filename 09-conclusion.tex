\section{Summary of Contributions}

This dissertation addresses a fundamental challenge in modern database systems:
enabling efficient timestamp-based concurrency control in disk-based
transactional databases. Through both theoretical analysis and experimental
evaluation, we demonstrate that approximate timestamp storage represents a
universal and future-ready approach for high-performance transaction processing
across diverse storage technologies.

Our work makes two primary contributions. First, we introduce \sketchname, a
novel approximate timestamp storage system that enables efficient
timestamp-based concurrency control in disk-based key-value stores. \sketchname
combines a hash table for exact timestamps of active keys with a probabilistic
sketch for approximate upper bounds of inactive keys, achieving the performance
benefits of fully in-memory timestamp storage while requiring only minimal
memory—as little as 32KB for an 80GB database. Second, we conduct a
comprehensive analytical and experimental study demonstrating that \sketchname
remains effective across a wide spectrum of storage technologies, from
traditional hard disk drives with millisecond latencies to emerging CXL-based
storage approaching DRAM-like speeds.

\subsection{The \sketchname System}

The first contribution establishes \sketchname as a practical solution to the
metadata storage problem in timestamp-based concurrency control. Our design is
grounded in a key insight: for timestamp-based protocols like STO, MVTO, and
TicToc, overapproximating timestamps does not violate correctness—it may cause
harmless extra aborts but preserves serializability. This insight allows us to
use approximate data structures while maintaining all correctness guarantees.

\sketchname's hybrid architecture, combining a foveated region (hash table) and
peripheral region (sketch), ensures that active keys maintain exact timestamps
throughout their transaction lifetime while inactive keys can be safely
approximated. We formally prove that this approach satisfies the necessary
properties for correctness with all three protocols we studied. Our evaluation
demonstrates that \sketchname-based implementations achieve dramatic performance
improvements: TicToc with \psketchname improves goodput by up to 5.9$\times$
over disk-based timestamp storage, up to 14$\times$ over traditional 2PL, and
reaches performance close to an idealized in-memory system.

\subsection{Broad Applicability Across Storage Technologies}

The second contribution establishes the universal applicability of the
approximate timestamp storage approach. Our comprehensive evaluation across
HDDs, SATA SSDs, NVMe SSDs, and emulated CXL-based storage reveals that
\sketchname's benefits scale with the fundamental gap between local memory and
remote storage access. This finding ensures that \sketchname will remain
valuable as storage technology continues to evolve.

On slow storage (HDDs and SATA SSDs), where disk I/O dominates performance,
\sketchname eliminates the prohibitive overhead of timestamp disk accesses. Our
results show improvements of up to 569\% on SATA SSD and 519\% on HDD compared
to disk-based timestamp storage in high-contention, write-intensive workloads.
Even in medium-contention scenarios where storage I/O is the primary bottleneck,
\sketchname provides substantial improvements.

On fast storage (emulated CXL-based flash), \sketchname transforms from
primarily avoiding disk I/O to functioning as an application-specific cache for
concurrency metadata. By keeping frequently accessed timestamps in local DRAM
rather than on remote memory-like storage, \sketchname prevents
microsecond-scale delays from accumulating into significant performance
penalties. TicToc-\psketchname achieves up to 3.55$\times$ and 21.5$\times$
higher goodput than 2PL and KR-OCC, respectively, on fast storage.

\begin{table}[!t]
\centering
\caption{Summary of \sketchname Performance Improvements Across Storage Technologies}
\label{tab:summary-performance}
\begin{tabular}{lccc}
\toprule
Storage Type & Best Improvement vs. Disk & Best Improvement vs. 2PL & Memory Usage \\
\midrule
HDD & 519\% (TicToc, high-contention) & 1.8$\times$ (TicToc, write-intensive) & 32KB for 80GB DB \\
SATA SSD & 569\% (TicToc, high-contention) & 5.89$\times$ (TicToc, write-intensive) & 32KB for 80GB DB \\
NVMe SSD & 5.9$\times$ (TicToc) & 14$\times$ (TicToc) & 32KB for 80GB DB \\
CXL (emulated) & N/A & 3.55$\times$ (TicToc-\psketchname vs. 2PL) & 32KB for 80GB DB \\
\midrule
\multicolumn{4}{l}{\footnotesize All improvements shown for TicToc-\psketchname unless otherwise noted.} \\
\multicolumn{4}{l}{\footnotesize Memory usage represents sketch size; hash table size varies with active keys.} \\
\bottomrule
\end{tabular}
\end{table}

\Cref{tab:summary-performance} provides a quantitative summary of \sketchname's
performance improvements across the storage technologies we evaluated. The table
demonstrates that \sketchname consistently delivers substantial performance gains
regardless of storage type, with the largest improvements occurring in
high-contention scenarios where timestamp access overhead is most pronounced.
Notably, the memory footprint remains constant at just 32KB for an 80GB database
across all storage technologies, demonstrating \sketchname's scalability.

\section{Key Findings and Insights}

Our research yields several important findings that guide both current
deployment and future development:

\subsection{Performance Characteristics}

Across all evaluated storage technologies and workloads, \psketchname
consistently outperforms both \fsketchname and disk-based timestamp storage. The
\psketchname variant achieves goodput that is often within 20-40\% of the
idealized Memory configuration while using only a tiny fraction of memory.
TicToc with \psketchname emerges as the clear performance leader, significantly
outperforming traditional concurrency control methods across all scenarios.

The memory efficiency of \sketchname is remarkable: a 32KB sketch suffices for
an 80GB database, representing less than 0.00004\% of the database size. This
efficiency makes \sketchname practical even in memory-constrained environments
where storing all timestamps in RAM would be infeasible.

\subsection{Storage-Dependent Behavior}

Our analysis reveals that the nature of \sketchname's advantages changes
fundamentally as storage performance improves. On slow storage, \sketchname
primarily eliminates I/O bottlenecks, achieving performance close to idealized
in-memory systems. On fast storage, while \sketchname still outperforms
disk-based approaches, CPU overhead from sketch operations becomes more visible,
creating optimization opportunities for future work.

Importantly, \sketchname remains beneficial across the entire storage spectrum
because it addresses the persistent gap between local computation and remote
storage access—a gap that exists regardless of whether storage is measured in
milliseconds or microseconds.

\subsection{Workload Sensitivity}

\sketchname demonstrates consistent effectiveness across diverse workloads, from
short transactions typical of OLTP systems to longer analytical queries.
High-contention, write-intensive workloads show the largest improvements, but
even medium-contention scenarios benefit substantially. Mixed workloads
containing both short and long transactions maintain \sketchname's
effectiveness, demonstrating its robustness.

\subsection{Protocol Compatibility}

Our formal analysis and experimental evaluation confirm that \sketchname
correctly integrates with STO, MVTO, and TicToc without any algorithmic changes
to these protocols. This plug-and-play compatibility makes adoption
straightforward and suggests that \sketchname could similarly integrate with
other timestamp-based protocols or future innovations in concurrency control.

\section{Broader Implications}

The success of \sketchname points to a broader principle: approximate metadata
management can enable high-performance system designs that would otherwise be
impractical. The key insight—that overapproximation preserves correctness for
many concurrency control protocols—may find application beyond timestamp
storage.

As storage technologies continue evolving toward faster, more memory-like
interfaces, the distinction between memory and storage blurs. \sketchname
demonstrates how application-specific caching strategies can bridge this gap,
providing a template for managing other types of frequently accessed metadata in
future systems.

Our work also highlights an important shift in database system design.
Traditional systems optimized for disk I/O as the dominant bottleneck. Modern
systems must optimize for CPU efficiency and memory hierarchy utilization.
\sketchname exemplifies this new design paradigm by prioritizing CPU and memory
efficiency while maintaining correctness guarantees.

\section{Future Work}

While \sketchname demonstrates strong performance across diverse storage
technologies, several directions offer opportunities for further improvement and
broader application.

\subsection{Optimizations for Fast Storage}

Our evaluation reveals that on fast storage (CXL-based or similar), \sketchname
incurs CPU overhead from sketch operations, memory allocation, and key
management that prevents it from fully matching the idealized Memory
configuration. Future work could explore several optimization strategies:

First, \emph{lock-free and wait-free algorithms} could reduce synchronization
overhead in the sketch and hash table operations. Current implementations use
per-bucket locks which, while correct, may become bottlenecks on fast storage
where operations complete in nanoseconds.

Second, \emph{custom memory allocators} optimized for the access patterns of
\sketchname could reduce allocation overhead. Fast storage environments are
CPU-bound, making efficient memory management critical.

Third, \emph{hardware acceleration} could leverage modern CPU features like SIMD
instructions for bulk sketch operations or hardware transactional memory for
lock-free updates. Exploring how emerging CPU architectures can accelerate
\sketchname operations presents an interesting research direction.

Finally, \emph{adaptive sketch sizing} could dynamically adjust sketch size
based on workload characteristics, reducing overhead for low-contention
scenarios while maintaining accuracy for high-contention workloads.

\subsection{Integration with Distributed Systems}

This dissertation focuses on single-node systems, but many production databases
operate in distributed environments. Extending \sketchname to distributed
systems raises several research questions:

How should timestamps be synchronized across nodes while maintaining
\sketchname's efficiency? Distributed timestamp allocation introduces
coordination overhead that could negate \sketchname's benefits. Research into
lightweight distributed timestamp protocols compatible with \sketchname would be
valuable.

Can each node maintain its own \sketchname instance for local keys? This
approach seems promising for shared-nothing architectures where transactions
primarily access local partitions. However, cross-partition transactions may
require different strategies.

How do network latencies affect the trade-offs that make \sketchname beneficial?
In distributed systems, network communication often dominates latency,
potentially changing the relative importance of local timestamp access
optimization.

\subsection{Advanced Concurrency Control Protocols}

While we demonstrate \sketchname's compatibility with STO, MVTO, and TicToc,
many other timestamp-based protocols exist. Investigating compatibility with
protocols like Serializable Snapshot Isolation (SSI), variations of OCC with
timestamps, or hybrid approaches could broaden \sketchname's applicability.

More fundamentally, could \sketchname inspire new concurrency control protocols
designed explicitly for approximate timestamp storage? The guarantees provided
by \sketchname may enable protocols that are not feasible with exact timestamps.

\subsection{Theoretical Analysis and Bounds}

Our heuristic analysis of sketch sizing provides practical guidelines, but
tighter theoretical bounds would strengthen the foundation. Can we provide
provable guarantees on abort rates given sketch size and workload
characteristics? Understanding the worst-case and expected-case behavior of
\sketchname under adversarial workloads would provide stronger guarantees.

The relationship between sketch size, contention level, transaction length, and
abort rates deserves deeper mathematical analysis. Such analysis could inform
automatic tuning mechanisms that adapt \sketchname parameters to workload
characteristics.

\subsection{Integration with Existing Database Systems}

While we evaluate \sketchname with SplinterDB, integration with other storage
systems would validate broader applicability. Integrating \sketchname with
LSM-tree based systems like RocksDB, B-tree systems like PostgreSQL, or other
storage architectures could reveal system-specific optimization opportunities.

Particularly interesting would be integration with systems that already maintain
some form of timestamp or version metadata, such as multi-version concurrency
control systems. Could \sketchname replace or enhance existing timestamp
management mechanisms?

\subsection{Other Storage Technologies}

Our evaluation covers HDD, SATA SSD, NVMe SSD, and emulated CXL-based storage,
but storage technology continues evolving. Evaluating \sketchname with emerging
technologies like 3D XPoint (Optane), persistent memory, or future memory-like
storage would validate its continued relevance.

As storage becomes faster and more memory-like, understanding the point at which
approximate timestamp storage becomes unnecessary would provide important design
guidance. This transition point may depend on workload characteristics, system
architecture, and economic factors.

\subsection{Beyond Timestamps: General Approximate Metadata}

The principle underlying \sketchname—that approximate metadata can preserve
correctness for certain protocols—may apply beyond timestamps. Could similar
techniques optimize storage of locks, version numbers, conflict detection
metadata, or other concurrency control state?

Exploring the space of metadata that can be safely approximated could yield
additional optimization opportunities. This direction requires identifying
metadata properties that allow approximation without violating correctness.

\subsection{Workload-Aware Optimization}

Our evaluation demonstrates that \sketchname's effectiveness varies with
workload characteristics. Future work could develop \emph{adaptive \sketchname}
variants that automatically adjust their behavior based on observed workload
patterns. For example, sketch size could adapt to contention levels, eviction
policies could optimize for observed access patterns, or the hash table could
resize based on active key counts.

Machine learning techniques could potentially optimize \sketchname parameters
based on historical workload data, automatically tuning for best performance
without manual intervention.

\section{Concluding Remarks}

This dissertation demonstrates that approximate timestamp storage enables
efficient timestamp-based concurrency control in disk-based databases while
requiring minimal memory. Through the design and evaluation of \sketchname, we
establish that this approach remains effective across a wide spectrum of storage
technologies, from traditional hard drives to emerging memory-like storage.

The key contribution of this work is not merely a new data structure, but a
demonstration that approximate metadata management can unlock high-performance
system designs that would otherwise be impractical. As storage technology
continues evolving and the gap between memory and storage narrows, techniques
like \sketchname that optimize metadata access will become increasingly
important for achieving optimal database performance.

\sketchname represents a practical, deployable solution to a fundamental
challenge in modern database systems. By requiring only 32KB of memory for an
80GB database while achieving performance close to idealized in-memory systems,
\sketchname makes advanced concurrency control protocols accessible to
real-world database deployments. The universal applicability of \sketchname
across storage technologies ensures its relevance both today and as storage
continues evolving.

More broadly, this work contributes to a paradigm shift in database system
design, from optimizing for disk I/O to optimizing for CPU efficiency and memory
hierarchy utilization. As this shift continues, approximate metadata management
techniques will play an increasingly central role in high-performance database
systems.

